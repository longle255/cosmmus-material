\newpage
\section{Summary}
Since Bitcoin's initial launch in 2009, blockchain technologies have attracted
extensive worldwide attention. Since then, several blockchain implementations
were proposed, each focusing on solving specific shortcomings of the original
blockchain proposal. Ethereum is after Bitcoin the second-largest cryptocurrency
platform by market capitalization. Unlike Bitcoin, Ethereum supports programmable
smart contracts with an entire programming language along with it, running in
Ethereum Virtual Machine (EVM) \cite{ethereum:evm}. The flexibility within a
Smart Contract enables users to implement their own autonomous execution logic
on blockchains \cite{delmolino2016step, buterin2014next, kosba2016hawk}. The
growing adoption of blockchain-based systems has raised many new challenges. In
particular, a primary and urgent concern in this context is to provide the level
of scalability to achieve a target throughput and latency in the presence of
increasing workload without compromising the decentralized nature of
blockchains. Currently, Bitcoin takes \emph{10 minutes} or longer to confirm a transaction,
with a maximum throughput of \emph{7 transactions per second}
\cite{nakamoto2019bitcoin}, while the number is \emph{15 seconds} and \emph{15
transactions per second} for Ethereum respectively \cite{ethereum:sharding}. 

One direction to scale the performance of blockchain systems is to applying
partitioning (e.g., \cite{facebookTAO, sciascia2012sdur, aguilera2007sinfonia}).
Over more than three decades, partitioning has been an area of interest and has
been studied in many domains, such as database systems, file systems, and
distributed object systems. This is challenging, however, due to the fundamental
difference in failure models between databases and blockchain. With its
decentralized nature, there is no single entity in a blockchain system contain
information of all objects in the network, which makes it difficult for
blockchain clients to locate objects in the presence of partitioning. In
addition, with the lacking of overall information, choosing a partition to put
the objects in order to maximize the balance between partitions is not
effective. Even if enough information is available, finding a good partitioning
is a complex optimization problem~\cite{curino2010sch,taft2014est}.

This project proposes an oracle service that provides information for
blockchain's smart contract to achieve optimize partitioning. The oracle service
consists of two main components, on-chain smart contract, and off-chain service.
The on-chain component is in the form of a smart contract, replies to the request of
user smart contract. The off-chain component is a centralized \lle{or even
decentralized} service that monitors the blockchain by analyzing the transaction
history. After receiving a request from user smart contract, the on-chain and
off-chain components communicate to extract the necessary information for the
request. With the hints from the oracle service, user smart contracts can
effectively choose a partition to place objects to maximize the balance of the
network, thus increase the transaction speed, as well as minimize the cost of
execution. In addition, unlike some other scaling proposals that require
extensive modification of the current blockchain code, our proposal will work
with most of the blockchains that support smart contract, without changing client
code.
