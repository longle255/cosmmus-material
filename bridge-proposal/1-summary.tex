%!TEX root =  main.tex
\newpage
\section{Summary}
Since Bitcoin's initial launch in 2009, blockchain technologies have attracted
extensive worldwide attention. Since then, several blockchain implementations
were proposed, each focusing on solving specific shortcomings of the original
blockchain proposal. Some implementations focus on special features, such as
smart contract support \cite{buterin2013ethereum, elrom2019neo}, privacy
\cite{alonso2018monero}. Some others allow enterprises to create private
blockchains, where only a defined set of entities can participate
\cite{androulaki2018hyperledger}. The growing adoption of blockchain-based
systems has raised many new challenges. In particular, two primary and urgent
challenges in this context are scalability and interoperability. Scalability is
the ability to achieve a target throughput and latency in the presence of
increasing workload without compromising the decentralized nature of
blockchains. Interoperability allows communication and data exchanging between
multiple different blockchains. 

Scalable performance could be achieve by state partitioning (e.g.,
\cite{facebookTAO, sciascia2012sdur, aguilera2007sinfonia}). State partitioning
is challenging, however, due to fundamental differences between previous domains
and blockchain, notably the failure model. With its decentralized nature, there
is no single entity in a blockchain system that contains information of all
objects in the network, which makes it difficult for blockchain clients to
locate objects in the presence of partitioning. In addition, with the lacking of
overall information, choosing a partition to place the objects in order to
maximize the balance between partitions is not effective. Even if enough
information is available, finding a good partitioning is a complex optimization
problem~\cite{curino2010sch,taft2014est}. Interoperability, on the other hand,
is gaining more attention in the blockchain community. Some blockchain systems
has introduced inter blockchain communication (IBC) protocols that enable
communication and data exchanging between multiple blockchains, allows users to
choose where to place smart contracts \cite{kwon2016cosmos, thomas2015protocol,
kokoris2018omniledger, al2017chainspace}. Moreover, some solutions start to
appear that allow smart contracts to move between blockchains
\cite{fynn2020move, back2014enabling, herlihy2018atomic}.

Scalability and interoperability are different problems of blockchain, they are
however in need a common component: an oracle that provides some information of
the blockchains. A partitioned blockchains needs an overview of the partitioning
in order to place the objects and contract on an optimized location to maintain
the balance of the network and therefore increase the performance. For
blockchains to interact with each other, some information of the other
blockchains like the cost of transaction, the performance of the network is also
necessary. This project proposes an oracle service that provides such
information. The oracle service consists of two main components, one on-chain
and one off-chain. The on-chain component is in the form of a smart contract,
replies to the request of user smart contract. The off-chain component is a
service that monitors the blockchains by analyzing the transaction history.
After receiving a request from user smart contract, the on-chain and off-chain
components communicate to extract the necessary information for the request.
User smart contract pays per-query fees to get the hints from the oracle
service. The cost of each query depends on the amount of information returned.
With the information provided by the oracle, the blockchain system can
effectively choose a partition to place objects to maximize the balance of the
network, thus increase the transaction speed, as well as optimizing the the
execution cost with cross-chain transaction.

% In addition, unlike some other scaling proposals that require
% extensive modification of the current blockchain code, our proposal will work
% with most of the blockchains that support smart contract, without changing
% client code.


% fee changes base on the hints