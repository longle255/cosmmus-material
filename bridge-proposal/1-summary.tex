%!TEX root =  main.tex
\newpage
\thispagestyle{empty}
\section{Summary}
Since Bitcoin's initial launch in 2009, blockchain technologies have attracted
extensive worldwide attention. Since then, several blockchain implementations
were proposed, each focusing on solving specific shortcomings of the original
blockchain proposal. Some implementations focus on special features, such as
smart contract support \cite{buterin2013ethereum, elrom2019neo}, privacy
\cite{alonso2018monero}. Some others allow enterprises to create private
blockchains, where only a defined set of entities can participate
\cite{androulaki2018hyperledger}. The growing adoption of blockchain-based
systems have raised many new challenges. Two primary challenges in this context
are scalability and interoperability. 

Scalability is the ability to achieve a target throughput and latency in the
presence of increasing workload without compromising the decentralized nature of
blockchains. Scalability could be achieve by state partitioning (e.g.,
\cite{facebookTAO, sciascia2012sdur, aguilera2007sinfonia}). It is crucial to
come up with a partitioning where most requests or transactions access one
partition only and are equally distributed among partitions. That way,
partitions can work in parallel to maximize the performance and improve the
throughput. Besides, when partitioning a blockchain, other factors need to be
considered: computation, storage, and bandwidth. On the other hand,
interoperability allows communication and data exchanged between multiple
different blockchains (e.g., sharing data between public and private
blockchains, or exchanging cryptocurrencies such as bitcoins for ether).
Interoperability is gaining more attention in the blockchain community. Some
blockchain systems have introduced inter blockchain communication (IBC)
protocols that enable communication and data exchanging between multiple
blockchains, allows users to choose where to place smart contracts
\cite{kwon2016cosmos, thomas2015protocol, kokoris2018omniledger,
al2017chainspace}. Moreover, some solutions start to appear that allows smart
contracts to move between existing blockchains \cite{fynn2020move, back2014enabling,
herlihy2018atomic}. 

Scalability and interoperability are different problems of blockchain; they are,
however, having the same concern. They both need a mechanism that tracks and
provides some information about the blockchains. A partitioned blockchain needs
an overview of the partitioning of the system to compute an optimized location
to place contracts and objects, therefore maintains the balance of the network
and increase the performance. In an inter blockchains transactions, where data
could be shared and exchanged between different blockchains, it's necessary to
track the movement of the data, as well as to have some information about the
parties like the cost of the transaction, the performance of the network, etc.

This project proposes an oracle service that provides such information. The
oracle service consists of two main components, one on-chain and one off-chain.
The on-chain component is in the form of a smart contract, replies to the
request of user smart contract. The off-chain component is a service that
monitors the blockchains by analyzing transaction history. After receiving a
request from user smart contract, the on-chain and off-chain components
communicate to extract the necessary information for the request. User smart
contract pays per-query fees to get the hints from the oracle service. The cost
of each query depends on the amount of information returned. With the
information provided by the oracle, the blockchain system can effectively choose
a partition to place objects to maximize the balance of the network, thus
increase the transaction speed, as well as optimizing the the execution cost
with cross-chain transactions.

% In addition, unlike some other scaling proposals that require
% extensive modification of the current blockchain code, our proposal will work
% with most of the blockchains that support smart contract, without changing
% client code.


% fee changes base on the hints