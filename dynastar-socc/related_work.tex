%!TEX root =  main.tex
\section{Related work}
\label{sec:rw}

State machine replication~\cite{Kapritsos:2012um, kotla2004htbft, Lam78, santos2013htsmr, Sch90} provides strong consistency guarantees, which come from total order and deterministic execution of commands.
%Deterministic execution is usually ensured by having every replica execute commands sequentially.
Since consistent ordering is fundamental for SMR, some authors proposed to optimize the ordering and propagation of commands.
% (i.e., the atomic broadcast layer of the system).
For instance, \cite{kapritsos2010scalable} proposes to divide the ordering of commands between different clusters: each cluster orders only some requests, and then forwards the partial order to every server replica, which then merges the partial orders deterministically into a single total order that is consistent across the system.
In~\cite{biely2012spaxos}, Paxos~\cite{Lamport:1998ea} is used to order commands, but it is implemented in a way that avoids overloading the leader process, which would turn it into a bottleneck.

Multi-threaded execution is a potential source of non-determinism, depending on how threads are scheduled to be executed in the operating system.
Some works attempted to circumvent this problems and come up with a multi-threaded, yet deterministic implementation of SMR.
In \cite{santos2013htsmr}, the authors propose to parallelize the receipt and dispatching of commands, while executing commands sequentially.
In \cite{kotla2004htbft}, application semantics is used to determine which commands can be executed concurrently and still produce a deterministic outcome (e.g., read-only commands).
In \cite{Kapritsos:2012um}, commands are tentatively executed in parallel.
After the parallel execution, replicas verify whether they reached a consistent state; if not, commands are rolled back and re-executed sequentially.

Many database replication schemes aim at achieving high throughput by relaxing consistency, that is, they do not ensure linearizability.
In deferred-update replication \cite{chundi96dur, kobus2013hybrid, sciascia2012sdur, SousaOMP01}, replicas commit read-only transactions immediately, not always synchronizing with each other.
Although this indeed improves performance, it allows non-linearizable executions.
Database systems usually ensure serializability \cite{BHG87} or snapshot isolation \cite{LinKJPA09}, which do not take into account real-time precedence of different commands among different clients. 
For some applications, these consistency levels may be enough, allowing the system to scale better, but services that require linearizability cannot be implemented with such techniques.

Efforts to make linearizable systems scalable have been made in the past~\cite{bezerra2014ssmr, corbett2013spanner, Glendenning2011, hoang2016, Marandi11}.
In \cite{Glendenning2011}, the authors propose a scalable key-value store based on DHTs, ensuring linearizability, but only for requests that access the same key. 
In \cite{Marandi11}, variant of SMR is proposed in which data items are partitioned but commands have to be totally ordered.
Spanner~\cite{corbett2013spanner} uses a separate Paxos group per partition and, to ensure strong consistency across partitions, clocks are assumed to be synchronized.
Although the authors say that Spanner works well with GPS and atomic clocks, if clocks become out of synch beyond tolerated bounds, correctness is not guaranteed.
\ssmr{}~\cite{bezerra2014ssmr} ensures consistency across partitions without any assumption about clock synchronization, but relies on a static partitioning of the state.
\dssmr{}~\cite{hoang2016} extends \ssmr\ by allowing state variables to migrate across partitions in order to reduce multi-partition commands.
However, \dssmr{} implements repartitioning in a very simple way that does not perform very well in scenarios with weak locality.
\dynastar\ improves on \dssmr\ by employing well-known graph partitioning techniques to decide where each variable should be.
Moreover, \dynastar\ dillutes the cost of repartitioning by moving variables on-demand, that is, only when they are accessed by some command.

Graph partitioning is an interesting problem with many proposed solutions~\cite{Abou-Rjeili:2006,kernighan1970efficient,hendrickson2000graph}.
In this work, we do not introduce a new graph partitioning solution, but instead we use a well-known one (METIS~\cite{Abou-Rjeili:2006}) to partition the state of a service implemented with state machine replication.
Similarly to \dynastar{}, Schism~\cite{curino2010sch} and Clay~\cite{SerafiniTEPAS16} also use graph-based partitioning to decide where to place data items in a transactional database.
In either case, not much detail is given about how to handle repartitioning dynamically without violating consistency.
Sword~\cite{quamar2013sword} is another graph-based dynamic repartitioning technique.
It uses a hypergraph partitioning algorithm to distribute rows of tables in a relational database across database shards.
Sword does not ensure linearizability and it is not clear how it implements repartitions without violating consistency.
E-Store~\cite{taft2014est} is yet another repartitioning proposal for transactional databases.
It repartitions data according to access patterns from the workload.
It strives to minimize the number of multi-partition accesses and is able to redistribute data items among partitions during execution.
E-Store assumes that all non-replicated tables form a tree-schema based on foreign key relationships.
This has the drawback of ruling out graph-structured schemas and \mbox{$m$-$n$} relationships.
\dynastar\ is a more general approach that works with any kind of relationship between data items, while also ensuring linearizability.

Some replication schemes are ``dynamic'' in that they allow the membership to be reconfigured during execution (e.g., \cite{birman2010dsr,dustdar2007soc,guessoum2003dar}).
For instance, a multicast layer based on Paxos can be reconfigured by adding or removing acceptors. 
These systems are dynamic in a way that is orthogonal to what \dynastar\ proposes.
%\dynastar\ consists of allowing the \emph{state partitioning}, that is, which state variables belong to which partition, to change dynamically.
%The greatest challenge that is addressed by \dynastar\ is how to provide such a solution, with a dynamic partitioning oracle, while ensuring a very strong level of consistency (linearizability), as variables are created, deleted, and moved across partitions, based on the access patterns of the workload.

