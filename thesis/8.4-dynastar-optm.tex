\subsection{Performance optimizations}
\label{sec:dynastar-optm}

In the algorithm presented in the previous section, clients always need to
involve the oracle, and the oracle dispatches every command to the partitions
for execution. Obviously, if every command involves the oracle, the system is
unlikely to scale, as the oracle will likely become a bottleneck. To address
this issue, clients are equipped with a location cache. Before submitting a
command to the oracle, the client checks its location cache. If the cache
contains the partition of the variables needed by the command, the client can
atomically multicast the command to the involved partition and thereby avoid
contacting the oracle.

The client still needs to contact the oracle in one of these situations: (a)~the
cache contains outdated information; or (b)~the command is a create, in
which case it must involve the oracle, as explained before. If the cache contains
outdated information, it may address a partition that does not have the
information of all the variables accessed by the command. In this case, the
addressed partition tells the client to retry the command. The client then
contacts the oracle and updates its cache with the oracle's response. Although
outdated cache information results in execution overhead, it is expected to
happen rarely since repartitioning is not frequent.
