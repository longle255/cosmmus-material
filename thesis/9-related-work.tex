%!TEX root =  main.tex
\chapter[Related work]{Related work}
\label{sec:rw}

In this chapter, we review related work on optimizing and scaling state machine
replication, partitioning application state, and optimize graph partitioning.

\paragraph{Consensus and state machine replication}
State machine replication (SMR) was first introduced by Leslie Lamport as an
example in ~\cite{Lam78}. Schneider \cite{Sch90} then gave a more systematic
approach to the design and implementation of SMR protocols. Util now SMR have
become a well-known approach to replication and has been extensively studied
~\cite{Kapritsos:2012um, Kotla:2004ep, santos2013htsmr}. SMR provides strong
consistency guarantees, which come from total order and deterministic execution
of commands. Deterministic execution is usually ensured by having every replica
execute commands sequentially. Traditional consensus based SMR repeatedly run
multiple instances of a consensus protocol to allow replicas to reach an agreed
order of commands. The best-known consensus algorithm is the Paxos protocol by
Lamport \cite{Lam98}. Raft \cite{184040}, a Paxos alternative also
implements consensus-based SMR, and it is suggested to be easier to understand
(than Paxos) from an engineering point of view.

\paragraph{Optimizing ordering protocol}

Even though increasing the performance of state machine replication is
non-trivial, different techniques have been proposed for achieving scalable
systems, such as optimizing the propagation and ordering of commands. This
allows the ordering layer (i.e., the underlying atomic broadcast algorithm) to
be itself also scalable. For instance, Kapritsos and
Junqueira~\cite{kapritsos2010scalable} propose to divide the ordering of
commands between different clusters: each cluster orders only some requests, and
then forwards the partial order to every server replica, which then merges the
partial orders deterministically into a single total order that is consistent
across the system. In S-Paxos~\cite{biely2012spaxos}, Paxos~\cite{Lam98} is used
to order commands, but it is implemented in a way such that the task of ordering
messages is evenly distributed among replicas, as opposed to having a leader
process that performs more work than the others and may eventually become a
bottleneck.

\paragraph{Parallelizing execution of commands}

Multi-threaded execution is a potential source of non-determinism, depending on
how threads are scheduled to be executed in the operating system. However, some
works have proposed multi-threaded implementations of state machine replication,
circumventing the non-determinism caused by concurrency in some way. In
\cite{santos2013htsmr}, for instance, the authors propose organizing each
replica in multiple modules that perform different tasks concurrently, such as
receiving messages, batching, and dispatching commands to be executed. The
execution of commands is still sequential, but the replica performs all other
tasks in parallel. In CBASE~\cite{Kotla:2004ep}, a parallelizer module uses
application semantics to determine which commands can be executed concurrently
without reducing determinism (e.g., read-only commands, which can be executed in
any order relative to one another). In Eve~\cite{Kapritsos:2012um}, commands are
tentatively executed in parallel. After the parallel execution, replicas verify
whether they reached a consistent state; if not, commands are rolled back and
re-executed sequentially.

\paragraph{Weakening consistency guarantees}

Many database replication schemes aim at achieving high throughput by relaxing
consistency, that is, they do not ensure linearizability. In deferred-update
replication \cite{chundi96dur, kobus2013hybrid, sciascia2012sdur, SousaOMP01},
replicas commit read-only transactions immediately, not always synchronizing
with each other. Although this indeed improves performance, it allows
non-linearizable executions. Database systems usually ensure serializability
\cite{BHG87} or snapshot isolation \cite{LinKJPA09}, which do not take into
account real-time precedence of different commands among different clients. For
some applications, these consistency levels may be enough, allowing the system
to scale better, but services that require linearizability cannot be implemented
with such techniques.

\paragraph{Partitioning application state}
Partitioning the state of a replicated service is conceptually similar to
partial replication of databases. Efforts to make linearizable systems scalable
have been made in the past~\cite{bezerra2014ssmr, corbett2013spanner,
Glendenning:2011kj, Marandi:2011dj}.  In Scatter~\cite{Glendenning:2011kj}, the
authors propose a scalable key-value store based on DHTs, ensuring
linearizability, but only for requests that access the same key. In the work of
Marandi et al~\cite{Marandi:2011dj}, variant of SMR is proposed in which data
items are partitioned but commands have to be totally ordered.
Spanner~\cite{corbett2013spanner} is a leader-leased, Paxos based system that
use TrueTime-accurate clock synchronization that require special hardwares to
improve geo-distributed read performance. Spanner uses a separate Paxos group
per partition and synchronized clocks to ensure strong consistency across
partitions. Although the authors say that Spanner works well with GPS and atomic
clocks, if clocks become out of synch beyond tolerated bounds, correctness is
not guaranteed. $M^2Paxos$~\cite{7579738} proposes a scheme where leases are
used instead of partitions owning objects, but assumes full state replication.
\ssmr{}~\cite{bezerra2014ssmr} ensures consistency across partitions without any
assumption about clock synchronization, but relies on a static partitioning of
the state. \dssmr{}~\cite{le2016dssmr} extends \ssmr\ by allowing state
variables to migrate across partitions in order to reduce multi-partition
commands. However, \dssmr{} implements repartitioning in a very simple way that
does not perform very well in scenarios where the state cannot be perfectly
partitioned. \dynastar\ improves on \dssmr\ by employing well-known graph
partitioning techniques to decide where each variable should be. Moreover,
\dynastar\ dilutes the cost of repartitioning by moving variables on-demand,
that is, only when they are accessed by some command.

\paragraph{Optimize graph partitioning}

Graph partitioning is an interesting problem with many proposed
solutions~\cite{Abou-Rjeili:2006,hendrickson2000graph,kernighan1970efficient,7004087}.
In this work, we do not introduce a new graph partitioning solution, but instead
we use a well-known one (METIS~\cite{Abou-Rjeili:2006}) to partition the state
of a service implemented with state machine replication. Similarly to
\dynastar{}, Schism~\cite{curino2010sch} and Clay~\cite{SerafiniTEPAS16} also
use graph-based partitioning to decide where to place data items in a
transactional database. In either case, not much detail is given about how to
handle repartitioning dynamically without violating consistency. Turcu et al.
~\cite{7004087} proposed a technique that reduces the amount of cross-partition
commands and implements an advanced transaction routing.
Sword~\cite{quamar2013sword} is another graph-based dynamic repartitioning
technique. It uses a hyper-graph partitioning algorithm to distribute rows of
tables in a relational database across database shards. Sword does not ensure
linearizability and it is not clear how it implements repartitioning without
violating consistency. E-Store~\cite{taft2014est} is yet another repartitioning
proposal for transactional databases. It repartitions data according to access
patterns from the workload. It strives to minimize the number of multi-partition
accesses and is able to redistribute data items among partitions during
execution. E-Store assumes that all non-replicated tables form a tree-schema
based on foreign key relationships. This has the drawback of ruling out
graph-structured schemas and \mbox{$m$-$n$} relationships. \dynastar\ is a more
general approach that works with any kind of relationship between data items,
while also ensuring linearizability.

Some replication schemes are ``dynamic'' in that they allow the membership to be
reconfigured during execution (e.g.,
\cite{birman2010dsr,dustdar2007soc,guessoum2003dar}). For instance, a multicast
layer based on Paxos can be reconfigured by adding or removing acceptors. These
systems are dynamic in a way that is orthogonal to what \dynastar\ proposes.

\dynastar\ consists of allowing the \emph{state partitioning}, that is, which
state variables belong to which partition, to change dynamically. The greatest
challenge that is addressed by \dynastar\ is how to provide such a solution,
with a dynamic partitioning oracle, while ensuring a very strong level of
consistency (linearizability), as variables are created, deleted, and moved
across partitions, based on the access patterns of the workload.

