%!TEX root =  main.tex
\chapter[Related work]{Related work}
\label{sec:rw}

In this chapter, we review some selected publications in the research areas considered
by this dissertation, specially in the context of state machine replication and scaling
the performance of state machine replication.

\section{State machine replication}

State machine replication (SMR) was first introduced by Leslie Lamport as an
example in ~\cite{Lam78}. Schneider \cite{Sch90} then gave a more systematic
approach to the design and implementation of SMR protocols. Util now, SMR has
become a well-known approach to replication and has been extensively studied,
both in academia ~\cite{Kapritsos:2012um, Kotla:2004ep, santos2013htsmr,
Mencius} and in the industry. SMR provides strong consistency guarantees, which
come from total order and deterministic execution of commands. Traditional SMR
relies on a consensus protocol to define a common order of execution of such
commands. Deterministic execution is usually ensured by having every replica
execute the set of ordered commands sequentially.


\section{Consensus and state machine replication}

Consensus \cite{Lam78, paprzycki:2001distributed} is a problem that requires one
or more processes to cooperate, each with an initial value, to eventually agree on
a single such value. It was proved that in any asynchronous system with faulty
processors, there is no deterministic algorithm providing termination
\cite{FLP85}. One solution to prevent processes in the asynchronous system from
not making progress is to use failure detectors \cite{aguilera:2000failure}.
Chandra and Toueg \cite{CT96} propose a class of algorithms that use failure
detectors to solve consensus. Another solution is to introduce synchrony and
assume a known delay to the system's communication \cite{Aspnes:2003vp}.

Traditional consensus-based SMR repeatedly runs multiple instances of a consensus
protocol to allow replicas to reach an agreed order of commands. The best-known
consensus algorithm is the Paxos protocol by Lamport \cite{Lam98}. However,
Paxos's description leaves many open design questions. Several works have argued
that Paxos is not an easy algorithm to implement \cite{paxoslive,
Kirsch:2008paxos}. Raft \cite{184040}, a Paxos alternative also implements
consensus-based SMR, and it is suggested to be easier to understand (than Paxos)
from an engineering point of view. Despite its complexity, several systems
have been using Paxos to provide various abstractions such as storage systems
\cite{Bolosky:20011paxos}, locking services \cite{Mike:2006chubby}, and
distributed databases \cite{Baker:2011megastore}. Google Chubby
\cite{Mike:2006chubby}, Google Spanner \cite{corbett2013spanner} are employing
some implementation of the Paxos algorithm to achieve consensus and also to
cater for replication. Zap \cite{Hunt:2010zoo}, the atomic broadcast protocol at
the core of ZooKeeper, is a modified version of Paxos with a focus on
high-performance.

\section{Scaling state machine replication}

Even though increasing the performance of state machine replication is
non-trivial, different techniques have been proposed for achieving scalable
systems, such as partitioning the application state, parallelizing execution of
commands, weakening consistency or optimizing the propagation and ordering of
commands. 

\subsubsection{Partitioning application state}

Partitioning the state of a replicated service is conceptually similar to
partial replication of databases. Efforts to make linearizable systems scalable
have been made in the past~\cite{bezerra2014ssmr, corbett2013spanner,
Glendenning:2011kj, Marandi:2011dj}.  In Scatter~\cite{Glendenning:2011kj}, the
authors propose a scalable key-value store based on DHTs, ensuring
linearizability, but only for requests that access the same key. In the work of
Marandi et al~\cite{Marandi:2011dj}, variant of SMR is proposed in which data
items are partitioned, but commands have to be totally ordered.
Spanner~\cite{corbett2013spanner} is a leader-leased, Paxos based system that
uses TrueTime-accurate clock synchronization that requires special hardwares to
improve geo-distributed read performance. Spanner uses a separate Paxos group
per partition and synchronized clocks to ensure strong consistency across
partitions. Although the authors say that Spanner works well with GPS and atomic
clocks, if clocks become out of synch beyond tolerated bounds, correctness is
not guaranteed. $M^2Paxos$~\cite{7579738} proposes a scheme where leases are
used instead of partitions owning objects, but assumes full state replication.
\ssmr{}~\cite{bezerra2014ssmr} ensures consistency across partitions without any
assumption about clock synchronization, but relies on a static partitioning of
the state. P-store \cite{Schiper:2010pstore} is a genuine partial replication
protocol for wide area networks. P-store divided replicas into groups and
partitioned data between the groups to ensure that all replicas in the same
group replicate the same data items. When a transaction is committing, it sends
a message to all involved replicas to validate. If a transaction is local to the
group, each replica can individually decide whether to commit or abort;
otherwise, if the transaction is global, all the involved replicas exchange
votes in order to decide the outcome of the transaction.

% \dssmr{}~\cite{le2016dssmr} extends \ssmr\ by allowing state
% variables to migrate across partitions in order to reduce multi-partition
% commands. However, \dssmr{} implements repartitioning in a very simple way that
% does not perform very well in scenarios where the state cannot be perfectly
% partitioned. \dynastar\ improves on \dssmr\ by employing well-known graph
% partitioning techniques to decide where each variable should be. Moreover,
% \dynastar\ dilutes the cost of repartitioning by moving variables on-demand,
% that is, only when they are accessed by some command.

\subsubsection{Optimizing ordering protocol}

Several works have focused on scaling performance of state machine replication
by improving the performance of the ordering protocol. This allows the ordering
layer (i.e., the underlying atomic broadcast algorithm) to be itself also
scalable. For instance, Kapritsos and Junqueira~\cite{kapritsos2010scalable}
propose to divide the ordering of commands between different clusters: each
cluster orders only some requests, and then forwards the partial order to every
server replica, which then merges the partial orders deterministically into a
single total order that is consistent across the system. In
S-Paxos~\cite{biely2012spaxos}, Paxos~\cite{Lam98} is used to order commands,
but it is implemented in a way such that the task of ordering messages is evenly
distributed among replicas, as opposed to having a leader process that performs
more work than the others and may eventually become a bottleneck. Mencius
\cite{Mencius} proposes a rotating leader protocol designed for Wide-Area
Networks, which improves throughput by distributing over multiple replicas the
load that is usually concentrated on the leader. On the other hand, Ring-Paxos
\cite{Marandi:2012hb} focus instead on achieving high network efficiency on fast
Gigabit LANs. 

\subsubsection{Parallelizing execution of commands}

Multi-threaded execution is a potential source of non-determinism, depending on
how threads are scheduled to be executed in the operating system. However, some
works have proposed multi-threaded implementations of state machine replication,
circumventing the non-determinism caused by concurrency in some way. In
\cite{santos2013htsmr}, for instance, the authors propose organizing each
replica in multiple modules that perform different tasks concurrently, such as
receiving messages, batching, and dispatching commands to be executed. The
execution of commands is still sequential, but the replica performs all other
tasks in parallel. In CBASE~\cite{Kotla:2004ep}, a parallelizer module uses
application semantics to determine which commands can be executed concurrently
without reducing determinism (e.g., read-only commands, which can be executed in
any order relative to one another). In Eve~\cite{Kapritsos:2012um}, commands are
tentatively executed in parallel. After the parallel execution, replicas verify
whether they reached a consistent state; if not, commands are rolled back and
re-executed sequentially. Storyboard in \cite{Kapitza:2010Storyboard} was
introduced as an approach that supports deterministic execution in
multi-threading environments. Storyboard uses a forecasting mechanism to
predicts an ordered sequence of locks across replicas based on
application-specific knowledge. Guo et al.~\cite{guo2014rex} proposed Rex, a
replicated state-machine framework for a multi-core system that uses an
execute-agree-follow strategy. In Rex, A so-called primary server receives
requests and processes those requests deterministically. The executions of
requests could be in parallel. Rex uses locks to synchronize the concurrent
access to a shared variable. The primary server periodically proposes the trace
for agreement to the other replicas to update.

\subsubsection{Weakening consistency guarantees}

Many replication schemes aim at achieving high throughput by relaxing
consistency; that is, they do not ensure linearizability. In deferred-update
replication \cite{chundi96dur, kobus2013hybrid, sciascia2012sdur, SousaOMP01},
replicas commit read-only transactions immediately, not always synchronized
with each other. Although this indeed improves performance, it allows
non-linearizable executions. Database systems usually ensure serializability
\cite{BHG87} or snapshot isolation \cite{LinKJPA09}, which do not take into
account real-time precedence of different commands among different clients. For
some applications, these consistency levels may be enough, allowing the system
to scale better, but services that require linearizability cannot be implemented
with such techniques. Some applications do not require strong consistency for
most of their operations, can weakening the consistency guarantees to eventual
consistency (e.g., Facebook's TAO \cite{facebookTAO}). Eventual consistency
\cite{Gustavsson:2002eventual} means that the correct replicas will eventually
converge to a common state, even though some intermediate states might diverge
to a certain extent. Some other systems try to combine the benefits of weak and
strong consistency models by supporting both.  In Gemini \cite{Li2012geo},
transactions that can execute under weak consistency run fast without needing
to coordinate with other datacenters. PNUTS \cite{Cooper2008PNUTSYH} and
DynamoDB \cite{Sivasubramanian:2012dynamo} also combine weak consistency with
per-object strong consistency. Both works rely on conditional writes, where a
write fails in the presence of concurrent writes.

\section{Graph partitioning in performance scaling}

Graph partitioning is an interesting problem with many proposed
solutions~\cite{Abou-Rjeili:2006,hendrickson2000graph,kernighan1970efficient,7004087}.
In this work, we do not introduce a new graph partitioning solution, but instead,
we use a well-known one (METIS~\cite{Abou-Rjeili:2006}) to partition the state
of a service implemented with state machine replication. Similarly to
\dynastar{}, Schism~\cite{curino2010sch} and Clay~\cite{SerafiniTEPAS16} also
use graph-based partitioning to decide where to place data items in a
transactional database. In either case, not much detail is given about how to
handle repartitioning dynamically without violating consistency. Turcu et al.
~\cite{7004087} proposed a technique that reduces the amount of cross-partition
commands and implements advanced transaction routing.
Sword~\cite{quamar2013sword} is another graph-based dynamic repartitioning
technique. It uses a hyper-graph partitioning algorithm to distribute rows of
tables in a relational database across database shards. Sword does not ensure
linearizability, and it is not clear how it implements repartitioning without
violating consistency. E-Store~\cite{taft2014est} is yet another repartitioning
proposal for transactional databases. It repartitions data according to access
patterns from the workload. It strives to minimize the number of multi-partition
accesses and is able to redistribute data items among partitions during
execution. E-Store assumes that all non-replicated tables form a tree-schema
based on foreign key relationships. This has the drawback of ruling out
graph-structured schemas and \mbox{$m$-$n$} relationships. \dynastar\ is a more
general approach that works with any kind of relationship between data items,
while also ensuring linearizability.

Some replication schemes are ``dynamic'' in that they allow the membership to be
reconfigured during execution (e.g.,
\cite{birman2010dsr,dustdar2007soc,guessoum2003dar}). For instance, a multicast
layer based on Paxos can be reconfigured by adding or removing acceptors. These
systems are dynamic in a way that is orthogonal to what \dynastar\ proposes.

\dynastar\ consists of allowing the \emph{state partitioning}, that is, which
state variables belong to which partition to change dynamically. The greatest
challenge that is addressed by \dynastar\ is how to provide such a solution,
with a dynamic partitioning oracle, while ensuring a very strong level of
consistency (linearizability), as variables are created, deleted, and moved
across partitions, based on the access patterns of the workload.