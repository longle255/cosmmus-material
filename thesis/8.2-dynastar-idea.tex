\section{\dynastar}
\subsection{General idea}
\label{sec:dynastar-idea}

\dynastar defines a dynamic mapping of application variables to partitions.
Application programmers may also define other granularity of data when mapping
application state to partitions. For example, in our social network application
(\S\ref{sec:imp:\dynastarappname}), each user (together with the information associated
with the user) is mapped to a partition; in our TPC-C implementation
(\S\ref{sec:imp:tpcc}), every district in a warehouse is mapped to a partition.
Such a mapping is managed by a partitioning oracle, which is handled as a
replicated partition. The oracle allows the mapping of variables to partitions
to be retrieved or changed during execution. To simplify the discussion, in
\S\ref{sec:dynastar-idea}--\ref{sec:dynastar-detailed} we initially assume
that every command involves the oracle. In \S\ref{sec:dynastar-optm}, we explain how
clients can use a cache to avoid the oracle in the execution of most commands.

Clients in \dynastar submit commands to the oracle and wait for the reply.
\dynastar\ supports three types of commands: $create(v)$ creates a new variable
$v$ and initially maps it to a partition defined by the oracle; $access(\omega)$
is an application command that reads and modifies variables in set $\omega
\subseteq \vvm$; and $delete(v)$ removes $v$ from the service state. The reply
from the oracle is called a $prophecy$, and usually consists of a set of tuples
$\langle v, \ppm \rangle$, meaning $v \in \ppm$, and a target partition $\ppm_d$
on which the command will be executed. The $prophecy$ could also tell the
clients if a command cannot be executed (e.g., it accesses variables that do not
exist). If the command can be executed, the client waits for the reply from the
target partition.

If a command $C$ accesses variables in $\omega$ on a single partition, the
oracle multicasts $C$ to that partition for execution. If the command accesses
variables on multiple partitions, the oracle multicasts a $global(\omega,\ppm_d,
C)$ command to the involved partitions to gather all variables in $\omega$ to
the target partition $\ppm_d$. After having all required variables, the target
partition executes command $C$, sends the reply to the client, and returns the
variables to their source.

The oracle also collects hints from clients and partitions to build up a
workload graph and monitors the changes in the graph.
%\dynastar models a service workload as a graph $G = (V, E)$, where
In the workload graph, vertices represent state variables and edges dependencies
between variables. An edge connects two variables in the graph if a command
accesses both of them.
%A location oracle builds the workload graph based on feedback from the clients
%or partitions, as commands are executed.
Periodically, the oracle computes a new optimized partitioning and sends the
partitioning plan to all partitions. Upon delivering the new partitioning, the
partitions exchange variables and update their state accordingly. \dynastar
relocates variables without blocking the execution of commands.

