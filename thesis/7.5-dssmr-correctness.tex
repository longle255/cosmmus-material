\subsection{Correctness}
\label{sec:dssmr-correctness}

In this section, we argue that \dssmr\ ensures termination and linearizability.
By ensuring termination, we mean that for every command $C$ issued by a correct
client, a reply to $C$ different than $retry$ is eventually received by the
client. This assumes that at least one oracle process is correct and that every
partition has at least one correct server. Given these constraints, the only
thing that could prevent a command from terminating would be an execution that
forced the client proxy to keep retrying a command. This problem is trivially
solved by falling back to \ssmr\ after a predefined number of retries: at a
certain point, the client proxy multicast the command to all server and oracle
processes, which execute the command as in \ssmr{}, i.e., with coordination
among all partitions and the oracle.

As for linearizability, we argue that, if every command in execution \ex\ of
\dssmr\ is delivered by atomic multicast and is \emph{execution atomic} (as
defined in~\cite{bezerra2014ssmr}), then \ex\ is linearizable. We denote the
order given by atomic multicast by relation $\prec$. Given any two messages
$m_1$ and $m_2$, ``$m_1 \prec m_2$'' means that there exists a process that
delivers both messages and $m_1$ is delivered before $m_2$, or there is some
message $m'$ such that $m_1 \prec m'$ and $m' \prec m_2$, which can be written
as \mbox{$m_1 \prec m' \prec m_2$}.
%\fxnote[draft]{use the phrase ``there exists a process that'' }
Also, for the purposes of this proof, we consider the oracle to be a partition,
as it also \amdel{}s and executes application commands.

Suppose, by means of contradiction, that there exist two commands $x$ and $y$,
where $x$ finishes before $y$ starts, but $y \prec x$ in the execution. There
are two possibilities to be considered: (i) $x$ and $y$ are delivered by the
same process $p$, or (ii) no process delivers both $x$ and $y$.

In case (i), at least one process $p$ delivers both $x$ and $y$. As $x$ finishes
before $y$ starts, then $p$ delivers $x$, then $y$. From the properties of
atomic multicast, and since each partition is mapped to a multicast group, no
process delivers $y$, then $x$. Therefore, we reach a contradiction in this
case.

In case (ii), if there were no other commands in \ex, then the execution of $x$
and $y$ could be done in any order, which would contradict the supposition that
$y \prec x$. Therefore, there are commands $z_1, ..., z_n$ with atomic order $y
\prec z_1 \prec \cdots \prec z_n \prec x$, where some process $p_0$ (of
partition $\ppm_0$) delivers $y$, then $z_1$; some process $p_1 \in \ppm_1$
delivers $z_1$, then $z_2$, and so on: process $p_i \in \ppm_i$ delivers
$z_{i}$, then $z_{i+1}$, where $1 \leq i < n$. Finally, process $p_n \in \ppm_n$
delivers $z_n$, then $x$.

Let $z_0 = y$ and let $atomic(i)$ be the following predicate: ``For every
process $p_i \in \ppm_i$, $p_i$ finishes executing $z_i$ only after some $p_0
\in \ppm_0$ started executing $z_0$.'' We now claim that $atomic(i)$ is true for
every $i$, where $0 \leq i \leq n$. We prove our claim by induction.

%\begin{itemize}

%\item
\emph{Basis ($i=0$)}: $atomic(0)$ is obviously true, as $p_0$ can only finish
executing $z_0$ after starting executing it.

%\item
\emph{Induction step}: If $atomic(i)$, then $atomic(i+1)$. \\
Proof: Command $z_{i+1}$ is multicast to both $\ppm_i$ and $\ppm_{i+1}$. Since
$z_{i+1}$ is execution atomic, before any $p_{i+1} \in \ppm_{i+1}$ finishes
executing $z_{i+1}$, some $p_i \in \ppm_i$ starts executing $z_{i+1}$. Since
$z_i \prec z_{i+1}$, every $p_i \in \ppm_i$ start executing $z_{i+1}$ only after
finishing the execution of $z_i$. As $atomic(i)$ is true, this will only happen
after some $p_0 \in \ppm_0$ started executing $z_0$.

%\end{itemize}

As $z_n \prec x$, for every $p_n \in \ppm_n$, $p_n$ executes command $x$ only
after the execution of $z_n$ at $p_n$ finishes. From the above claim, this
happens only after some $p_0 \in \ppm_0$ starts executing $y$. This means that
$y$ ($z_0$) was issued by a client before any client received a response for
$x$, which contradicts the assumption that $x$ precedes $y$ in real-time, i.e.,
that command $y$ was issued after the reply for command $x$ was received.

