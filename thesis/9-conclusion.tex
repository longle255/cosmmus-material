\chapter[Conclusion]{Conclusion}


We have proposed here \dynastar, a novel state-machine replication approach that
allows state-machine replication to scale by dynamically adapting to the
workload, while still ensuring strong consistency. This is a work in progress,
though, and thus needs to be further elaborated. For this purpose, a few points
need to be addressed:

\begin{itemize}

    \item[i)]\emph{Decentralized partitioning:}
    Even though defining optimized partitioning by using a centralized oracle
    shows its advantages over the decentralized approach, the oracle is still
    prone to becoming a bottleneck as the workload becomes bigger or less
    clustered, since it has to handle more queries from clients. A decentralized
    graph partitioning approach could help solve this problem.

    \item[ii)]\emph{Reconfiguring partitions:}
    In \dynastar, we considered a fixed number of partitions. Changing the
    number of partitions in a \dynastar\ deployment while the system is running,
    is not a trivial problem to solve efficiently. Ideally \dynastar\ could
    allow application's state to shrink or expand to fit into a partitioning
    configuration with more or less partitions, in order to add or remove a
    partition dynamically.

\end{itemize}

Furthermore, \dynastar currently requires clients to explicitly send specific
commands to create workload graph (e.g., creating vertices or edges between
vertices). We plan to create a development environment, in the form of a
programming library, that allows the application designer to focus on the
application logic, rather than on graph partitioning and remote execution
details. The creation of the graph, handling of state partitioning and
distributed execution of commands will be handled internally by the library.
