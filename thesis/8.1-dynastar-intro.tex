
\dssmr{} addresses the limitations of \ssmr{} by adapting the partitioning
scheme as workloads change, by moving data ``on demand'' to maximize the number
of single partition user commands, while avoiding imbalances in the load of the
partitions. The major challenge in this approach is determining how the system
selects the partition to which to move data. \dssmr{} selects partitions
randomly, which allows for a completely decentralized implementation, in which
partitions make only local choices about data movement. We refer to this
approach as \emph{decentralized partitioning}. This approach works well for data
with \emph{strong locality}, but it is unstable for workloads with \emph{weak
locality}\footnote{Workloads are referred as \emph{strong locality} if that can
be partitioned in a way that would both (i) allow commands to be executed at a
single partition only, and (ii) evenly distribute data so that load is balanced
among partitions. Conversely, workloads that cannot avoid multi-partition
commands with balanced load among partitions exhibit \emph{weak locality}.}.
This happens because with weak locality, objects in \dssmr{} are constantly
being moved back and forth between partitions without converging to a stable
configuration.

In this chapter, we introduce \dynastar, a new dynamic approach to the state
partitioning problem. Like the other dynamic approach, \dynastar does not
require any a priori knowledge about the workload. However, \dynastar differs
from the prior approach because it maintains a location oracle with a global
view of the application state. The oracle minimizes the number of state
relocations by monitoring the workload, and re-computing an optimized
partitioning on demand using a static partitioning algorithm.

The location oracle maintains two data structures: (i) a mapping of application
state variables to partitions, and (ii) a \emph{workload graph} with state
variables as vertices and edges as commands that access the variables. Before a
client submits a command, it contacts the location oracle to discover the
partitions on which state variables are stored.  If the command accesses
variables in multiple partitions, the oracle chooses one partition to execute
the command and instructs the other involved partitions to temporarily relocate
the needed variables to the chosen partition. Of course, when relocating a
variable, the oracle is faced with a choice of which partition to use as a
destination. \dynastar chooses the partition for relocation by partitioning the
workload graph using the METIS~\cite{Abou-Rjeili:2006} partitioner and selecting
the partition that would minimize the number of state relocations.
% Although METIS does not necessarily produce the best possible partitioning of the workload graph, it offers a good compromise between performance and partitioning quality.

To tolerate failures, \dynastar implements the oracle as a regular partition,
replicated in a set of nodes. To ensure that the oracle does not become a
performance bottleneck, clients cache location information. Therefore, clients
only query the oracle when they access a variable for the first time or when
cached entries become invalid (i.e., because a variable changed location).
\dynastar copes with commands addressed to wrong partitions by telling clients
to retry a command.

We implemented \dynastar and compared its performance to other schemes,
including an idealized approach that knows the workload ahead of time. Although
this scheme is not achievable in practice, it provides an interesting baseline
to compare against. \dynastar's performance rivals that of the idealized scheme,
while having no a priori knowledge of the workload. We show that \dynastar
largely outperforms existing dynamic schemes under two representative workloads,
the TPC-C benchmark and a social network service. We also show that the location
oracle never becomes a bottleneck and can handle workload graphs with millions
of vertices.

In summary, this chapter makes the following contributions:
\begin{itemize}
\item It introduces \dynastar and discusses its implementation.
\item It evaluates different partitioning schemes for state machine replication
under a variety of conditions.
\item It presents a detailed experimental evaluation of \dynastar using the
TPC-C benchmark and a social network service populated with a real social
network graph that contains half a million users and 14 million edges.
\end{itemize}


The rest of the chapter is structured as follows.
Section~\ref{sec:dynastar-idea} introduces \dynastar and describes the technique in detail.
Section~\ref{sec:dynastar-algo} overviews our prototype.
Section~\ref{sec:dynastar-experiments} reports on the results of our experiments.
Section~\ref{sec:dynastar-conclusion} concludes the chapter.

