\chapter[System model and definitions]{System model and definitions}
\label{sec:sysmodel}

In this chapter, we present the system model, introduce two variations of a
multicast communication primitive, and define our correctness criterion (i.e.,
linearizability).

\section{System model}

Unless mentioned otherwise, we make the following assumptions about processes,
failures and system synchrony.


\begin{itemize}
  \item \textbf{Processes and communication.} We consider a distributed system
  consisting of an unbounded set of client processes $\ccm = \{c_1, c_2, ...\}$
  and a bounded set of server processes (replicas) $\ssm = \{s_1, ..., s_n\}$.
  Set $\ssm$ is divided into disjoint groups of servers $\ssm_0, ..., \ssm_k$.
  Processes communicate by message passing, using either one-to-one or
  one-to-many communication. One-to-one communication uses primitives
  $send(p,m)$ and $receive(m)$, where $m$ is a message and $p$ is the process
  $m$ is addressed to. If sender and receiver are correct, then every message
  sent is eventually received. One-to-many communication relies on reliable
  multicast and atomic multicast, defined in \S\ref{sec:rmcast} and
  \S\ref{sec:amcast}, respectively.
  \item \textbf{Failure model.} We assume the \emph{crash} failure model.
  Processes are either \emph{correct}, if they never fail, or \emph{faulty},
  otherwise. In either case, processes do not experience arbitrary behavior
  (i.e., no Byzantine failures).
  \item \textbf{Synchrony model.} Consensus cannot be solved deterministically
  in an asynchronous system with faults, as established by the FLP impossibility
  result~\cite{FLP85}. So, we consider a system that is \emph{partially
  synchronous}~\cite{DLS88}: the system is initially asynchronous and eventually becomes
  synchronous. When the system is asynchronous, there are no bounds on the time
  it takes for messages to be transmitted and actions to be executed; when the
  system  is synchronous, such bounds exist but are unknown to the processes.
  The partially synchronous assumption allows consensus defined in
  \S\ref{sec:consensus}, to be implemented under realistic
  conditions~\cite{FLP85,Lam98}.
\end{itemize}

\subsection{Reliable multicast}
\label{sec:rmcast}
\fxnote{Correcting multicast primitive}
\lle{Our protocol uses a uniform reliable multicast primitive 
that allows to multicast messages reliably to a subset of the groups in $\ssm$.
% Reliable broadcast provides stronger properties than regular broadcast, by
% ensuring that a message is either delivered to all correct processes or to none.
For each message $m$, $\gamma$ denotes the groups to which the message is reliably multicast.
By abuse of notation, we write $p \in \gamma$ instead of $\exists g \in \ssm : g \in \gamma \wedge p \in g$}
To reliably multicast a message $m$ to a set of groups $\gamma$ in $\ssm$, processes use
primitive \rmcast$(\gamma, m)$.  Message $m$ is delivered at the destinations
with \rmdel$(m)$.  Reliable multicast has the following properties:

\begin{itemize}

    \item[--] If a correct process \rmcast{}s $m$, then \lle{at least one
      correct process} $p \in \gamma$ eventually \rmdel{}s $m$ \emph{(validity)}.

    \item[--] If a correct process $p \in \gamma$ \rmdel{}s $m$, then every correct
      process $p \in \gamma$ eventually \rmdel{}s $m$ \emph{(uniform agreement)}.

    \item[--] For any message $m$, every process $p \in \gamma$ \rmdel{}s $m$
      at most once, and only if some process has \rmcast{} $m$ to $\gamma$
      previously \emph{(uniform integrity)}.

\end{itemize}

\subsection{Atomic multicast}
\label{sec:amcast}
Atomic multicast allows messages to be addressed to a subset of the processes in
the system. Atomic multicast ensures that the destination processes of every
message agree either to deliver or to not deliver the message, and no two
processes deliver any two messages in different order. To atomically multicast a
message $m$ to a set of groups $\gamma$, processes use primitive
\amcast$(\gamma, m)$.  Message $m$ is delivered at the destinations with
\amdel$(m)$.  We define delivery order $<$ as follows: $m < m'$ iff there exists
a process that delivers $m$ before $m'$.

Atomic multicast ensures the following properties:

\begin{itemize}

    \item[--] If a correct process \amcast{}s $m$ to group $\gamma$, every
    correct process in $\gamma$ \amdel{}s $m$ \emph{(validity)}.

    \item[--] If a process in $\gamma$ \amdel{}s $m$, then every correct process
    in $\gamma$ \amdel{}s $m$ \emph{(uniform agreement)}.

    \item[--] For any message $m$, every process \amdel{}s $m$ at most once, and
      only if some process has \amcast{} $m$ previously \emph{(integrity)}.

    \item[--] If a process \amcast{}s $m$ and then $m'$ to group $\gamma$, then
    no process in $\gamma$ \amdel{}s $m'$ before $m$ \emph{(fifo order)}.

    \item[--] The delivery order is acyclic \emph{(atomic order)}.

    \item[--] For any messages $m$ and $m'$ and any processes $p$ and $q$ such
      that $p \in g$, $q \in h$ and $\{ g, h \} \subseteq \gamma$, if $p$
      delivers $m$ and $q$ delivers $m'$, then either $p$ delivers $m'$ before
      $m$ or $q$ delivers $m$ before $m'$ \emph{(prefix order)}.

\end{itemize}

Atomic broadcast is a special case of atomic multicast in which there
is a single group of processes.

\section{Consensus}
\label{sec:consensus}
Consensus is a fundamental coordination problem of distributed computing
\cite{Lam78, santos2013htsmr}. The problem is related to replication and appears
when implementing atomic broadcast, group membership, or similar services. Given
a set of servers proposing values, it is the problem of deciding on one value
among the servers. The consensus problem is defined by the primitives
\emph{propose(v)} and \emph{decide(v)}, where \emph{v} is an arbitrary value.
Any uniform consensus must satisfy the following three properties:
\begin{itemize}

  \item[--] If a process decides \emph{v}, then \emph{v} was previously
  proposed by some process \emph{(uniform integrity)}.

  \item[--] If one or more correct processes propose a value then eventually
  some value is decided by all correct processes \emph{(termination)}.

  \item[--] No two processes decide different values \emph{(uniform agreement)}.

\end{itemize}

\section{Consistency}
An object that can be concurrently accessed by many processes is called a
concurrent object \cite{linearizability}. Interleaving accesses to the same
object can sometimes lead to unexpected behaviors. The effect of this issue can
be captured by defining a consistency criterion over the shared object, which
specifies how operations can interleave in accessing the object.
Informally, systems that provide strong consistency are easier to interact with,
because they behave in an intuitive way: they behave as the system had only one
copy of the object, and all operations modified and read that object atomically.
In this thesis, we specifically focus on strong consistency.

\fxnote{adding citation for linearizability}
\fxnote{adding non-blocking property for linearizability}
Strong consistency commonly refers to the formal concept of either strict
serializability or linearizability~\cite{linearizability}. A system is strictly
serializable if the outcome of any sequence of operations, as observed by its
clients, is equivalent to a serialization of those operations in which the
temporal ordering of non-overlapping operations is respected (i.e., if an
operation \emph{$o_1$} is acknowledged by the system before another operation
\emph{$o_2$} is proposed by some clients, then \emph{$o_1$} comes before
\emph{$o_2$} in the equivalent serialization). Linearizability is a sub-case of
strict serializability in which every operation reads or updates a single
object. \lle{Linearizability is a ``local'' and ``non-blocking'' property. 
Linearizability is local in that it is sufficient for a system
to linearize operations for each individual object to achieve global
linearizability. Linearizability is non-blocking in that pending
operations do not have to wait for other pending operations to complete.}

Linearizability is defined with respect to a sequential
specification. The \emph{sequential specification} of a service consists of a
set of commands and a set of \emph{legal sequences of commands}, which define
the behavior of the service when it is accessed sequentially. In a legal
sequence of commands, every response to the invocation of a command immediately
follows its invocation, with no other invocation or response in between them.
For example, a sequence of operations for a read-write variable $v$ is legal if
every read command returns the value of the most recent write command that
precedes the read, if there is one, or the initial value, otherwise. An
execution \ex\ is linearizable if there is some permutation of the commands
executed in \ex\ that respects (i)~the service's sequential specification and
(ii)~the real-time precedence of commands. Command $C_1$ precedes command $C_2$
in real-time if the response of $C_1$ occurs before the invocation of $C_2$.


\section{State machine replication}
\label{sec:smr}
State machine replication, also called active replication, is a common approach
to building fault-tolerant systems~\cite{Lam78, Sch90}. State machines model
deterministic applications. They atomically execute commands issued by clients.
This results in a modification of the internal state of the state machine and
also in the production of an output to a client. An execution of a state machine
is completely determined by the sequence of commands it executes and is
independent of external inputs such as timeouts. A fault-tolerant state machine
can be implemented by replicating it over multiple servers. Commands must be
executed by every replica in a consistent order, despite the fact that different
replicas might receive them in different orders. To guarantee that servers
deliver the same sequence of commands, SMR can be implemented with atomic
broadcast: commands are atomically broadcast to all servers and all correct
servers deliver and execute the same sequence of commands \cite{BJ87b,DSU04}. In
this thesis, we consider implementations of state machine replication that
ensure linearizability.

\section{Scaling state machine replication}

State machine replication yields configurable availability but limited
scalability. Adding resources (i.e., replicas) results in a service that
tolerates more failures, but does not translate into sustainable improvements in
throughput. This happens for a couple reasons. First, the underlying
communication protocol needed to ensure ordered message delivery may not scale
itself (i.e., a communication bottleneck). Second, every command must be
executed sequentially by each replica (i.e., an execution bottleneck).

Several approaches have been proposed to address SMR's scalability limitations.
To cope with communication overhead, some proposals have suggested to spread the
load of ordering commands among multiple processes (e.g.,
\cite{Moraru:2013gw,Mencius,Marandi:2012hb}), as opposed to dedicating a single
process to determine the order of commands (e.g.,
\cite{Lam98}).%\fxnote[draft]{remove inapplicable reference}

Two directions of research have been suggested to overcome execution
bottlenecks. One approach (scaling up) is to take advantage of multiple cores to
execute commands concurrently without sacrificing consistency
\cite{Kapritsos:2012um,Marandi:2014bj,Kotla:2004ep,guo2014rex}. Ideally, one
could use a replication technique that supports parallelism (multithreading) to
scale up a replicated service. But existing techniques have at least one
sequential part in their execution. Another approach (scaling out) is to
partition the service's state (also known as \emph{sharding}) and replicate each
partition (e.g., \cite{Glendenning:2011kj,Marandi:2011dj}). The idea is to
divide the state of a service in multiple partitions so that most commands
access one partition only and are equally distributed among partitions.
Unfortunately, most services cannot be ``perfectly partitioned,'' that is, the
service state cannot be divided in a way that commands access one partition
only. As a consequence, partitioned systems must cope with multi-partition
commands. In the next chapter, we review some approaches in the second category,
which include Scalable State Machine Replication (\ssmr) \cite{bezerra2014ssmr},
Google Spanner \cite{corbett2013spanner}, and some other techniques.


\section{Graph partitioning problem}

In computer systems, graphs are widely used to describe the dependencies of the
data within a computation. All applications can model their data and/or workload
using a graph. For example, in a social network application, the workload could be
modeled as a weighted graph, the users are represented by nodes and the
connections/relations between those users by the edges between nodes. Once the
workload is modeled as a graph, graph partitioning can be used to
determine how to divide the work and data for efficient concurrent access or
computation.
\fxnote{Rewriting sentence}
A weighted (directed) graph $G$ consists of a set of nodes $V$ and
a set of edges $E \subset V \times V$ to represent relations between the nodes,
as well as two cost functions. One function assigns weights to the nodes $c : V
\rightarrow R_{>0}$ and a second function $\omega : E \rightarrow R$ assigns
costs to the edges. A subgraph is a graph whose
node and edge set are subsets of another graph. An edge that connects vertices
of two different subgraph is called a \emph{crossing edge}. Graph partitioning
is a fundamental problem of finding a specified number of distinct subsets of
the set of vertices of a graph (subgraph). \emph{Balance} is the constraint that
requires that all subgraph have about equal size. \emph{Total edge cut} is the
number of edges crossing those subgraphs. \lle{Intuitively, a good partitioning is
the one that minimizes the number of the total edges connecting subgraphs}, while maintaining
the balance among partitions. This is generally considered an NP-hard
problem \cite{karypis1998multilevelk}

\section{Workload locality}
\fxnote{Rewriting sentence}
\label{sysmodel:locality}
In order to achieve a good partitioning, it is necessary to exploit the locality
of a workload. Locality is one of the long-known and much-exploited principles in
computer science. Operating system and databases have been exploiting the
locality for years \cite{denning2006locality}. \lle{In the scope of this thesis, we define
the locality of a workload as the data access patterns
produced by the application. A workload has \emph{strong
locality}} if it can be partitioned in a way that would both (i) allow commands
to be executed at a single partition only, and (ii) evenly distribute data so
that load is balanced among partitions. Conversely, workloads that cannot avoid
multi-partition commands with balanced load among partitions exhibit \emph{weak
locality}.
