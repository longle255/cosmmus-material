\chapter[Introduction]{Introduction}

\section{Context and goal of the thesis}

Modern applications today deploy services across a massive and
continuously expanding infrastructure, in order to serve users efficiently
and reliably. With the wide availability of commodity hardware, failures in data
centers are increasingly common and inevitable, even in well-established service
providers \cite{disruption:google}. The causes of failures are varied. Machines
fail individually due to power failures, hardware failures, or collectively due
to human errors \cite{disruption:amazon}, software bugs, or even extreme weather
events \cite{disruption:weather}. While confronting such failures, applications
still need to offer uninterrupted service to their users. To this end,
introducing fault tolerance through redundancy is critical for the availability
and integrity of the applications.

Redundancy by replication can increase both availability and scalability. By
having several copies of the application running on multiple replicas, the
failure of one replica does not result in the interruption of the service.
Moreover, requests from users can be distributed across multiple replicas, so
the workload can be divided among different machines, which would translate into
better scalability. A main difficulty with replicating an application though is
managing consistency among replicas. A strong consistency system gives
programmers an intuitive model for the effects of concurrent updates (i.e.,
clients perform concurrent updates as there is a single copy of the application
state), making it less likely that complex system behavior will result in bugs.

State machine replication is a well-known form of software replication, in
particular of \emph{active replication}, to build fault-tolerant services using
commodity hardware (e.g.,
\cite{shvachko2010hadoop,ghemawat2003google,burrows2006chubby,maccormick2004boxwood}). In essence,
the idea is to model the application as a deterministic state machine whose
state transitions consist of the execution of client requests \cite{Lam78,
Sch90}. Then, the service is fully replicated on several servers that
deterministically execute every client command in the
same order to reach the same results. State machine replication provides
configurable fault tolerance in the sense that the system can be set to tolerate
any number of faulty replicas. In terms of scalability, unfortunately, classic
state machine replication does not scale. Increasing the number of replicas will
not scale performance since each replica must execute every command. Thus, the
overall performance is limited by the throughput of a single replica.

Conceptually, scalable performance can be achieved with state partitioning, also
known as sharding (e.g., \cite{facebookTAO, sciascia2012sdur, aguilera2007sinfonia}),
which partitions the persistent state of the service across multiple servers.
Ideally, if the service state (e.g., state variables) can be divided
such that commands access one partition only and are equally distributed among
partitions, then system throughput (i.e., the number of commands that can be
executed per time unit) will increase linearly with the number of partitions.
Although promising, exploiting partitioning in SMR is challenging. First, most
applications cannot be partitioned in such a way that commands always fall
within a single partition. Therefore, a partitioning scheme must cope with
multi-partition commands. Second, determining an efficient partitioning of the
state that avoids load imbalance and favors single-partition commands normally
is computationally expensive and requires an accurate characterization of the
workload. Even if enough information is available, finding a good partitioning
is a complex optimization problem~\cite{curino2010sch,taft2014est}. Third, many
online applications experience variations in demand. These happen for a number
of reasons. In social networks, some users may experience a surge increase in
their number of followers (e.g., new ``celebrities''); workload demand may shift
along the hours of the day and the days of the week, and unexpected (e.g., a
video that goes viral) or planned events (e.g., a new company starts trading in
the stock exchange) may lead to exceptional periods when requests increase
significantly higher than in normal periods. %Thus, an ideal partitioning scheme
% should be able
%to adapt to the changes in the workload dynamically without interruption of the
%service.

It is crucial that highly available partitioned systems be able to dynamically
adapt to the workload in an optimized way (e.g., minimize cross-partition
communication). We believe there should be a way to provide strong consistency
in a scalable manner, that is, to create a linearizable, scalable and efficient
replicated service. Throughout this thesis, we aim to solve the problems
mentioned above, while substantiating the following claim:

\begin{quote}
``It is possible to devise an approach that allows a fault-tolerant system to
scale by dynamically adapting to the changes in the workload, while providing
strong consistency, that is, ensuring linearizability.''
\end{quote}


\section{Research contributions}
\label{sec:contribution}


The contribution of this thesis is centered on scaling performance of a
replicated state machine system. In this section, we outline the main
contributions of this work and provide a short description of each one. We defer
detailed discussions to the next chapters.

% To achieve the goal defined above for the doctoral thesis, we first present a
% survey of current approaches on scaling a partitioned replicated system, by applying an abstraction model 
% surveying a set of In this study we considered a set of techniques
% that...\fxnote{add description of the evaluation chapter}

% With the understanding of the limitations of current approaches, 
To achieve the goal defined above for the doctoral thesis, we first conducted a
survey of different classes of techniques to partial replication. In particular, we extended the replication
framework introduced in \cite{pedone:replication} to account for partial replication.
We then proposed \textbf{\dssmrlong}\ (\dssmr), a technique that allows a
partitioned SMR system to reconfigure its data placement on-the-fly. \dssmr\
achieves dynamic data reconfiguration without sacrificing scalability or
violating the properties of classical SMR. These requirements introduce
significant challenges. Since state variables may change location, clients must
find the current location of variables. The scalability requirement rules out
the use of a centralized oracle that clients can consult to find out the
partitions a command must be multicast to. Even if clients can determine the
current location of the variables needed to execute a command, by the time the
command is delivered at the involved partitions, one or more variables may have
changed their location. Although the client can retry the command with the new
locations, how to guarantee that the command will succeed in the second attempt?
In classical SMR, every command invoked by a non-faulty client always succeeds.
\dssmr\ should provide similar guarantees. \dssmr\ was designed to exploit
workload locality. This scheme benefits from simple manifestations of locality,
such as commands that repeatedly access the same state variables, and more
complex cases, such as structural locality in social network applications, where
users with common interests have a higher probability of being interconnected in
the social graph. Focusing on locality allows us to adopt a simple but effective
approach to state reconfiguration: whenever a command requires data from
multiple partitions, the variables involved are moved to a single partition and
the command is executed against this partition. To reduce the chances of skewed
load among partitions, the destination partition is chosen randomly. Although
\dssmr\ could use more sophisticated forms of partitioning, formulated as an
optimization problem (e.g., \cite{curino2010sch,taft2014est}), it has
the advantage that it does not need any prior information about the workload and
is not computationally expensive.

\dssmr{} addresses the limitations of static approaches by adapting the partitioning
scheme as workloads change, by moving data ``on demand'' to maximize the number
of single partition user commands, while avoiding imbalances in the load of the
partitions. The major challenge in this approach is determining how the system
selects the partition to which to move data. \dssmr{} selects partitions
randomly, which allows for a completely decentralized implementation, in which
partitions make only local choices about data movement. We refer to this
approach as \emph{decentralized partitioning}. This approach works well for data
with \emph{strong locality}, but it is unstable for workloads with \emph{weak
locality}.\footnote{The definition of \emph{strong-} and \emph{weak-locality} is
postponed until Section \ref{sysmodel:locality}.} This happens because with
weak locality, objects in \dssmr{} are constantly being moved back and forth
between partitions without converging to a stable configuration.

For this reason, we introduce the main contribution of this thesis,
\textbf{\dynastar, Optimized Dynamic Partitioning for Scalable State Machine
Replication}. Like DS-SMR, \dynastar does not require any a priori knowledge
about the workload. However, \dynastar differs from the prior approach because
it creates the workload graph on-the-fly and uses graph partitioning techniques
to efficiently relocate application state on-demand. In order to reach an
optimized partitioning,  \dynastar maintains a location oracle with a global
view of the application state. The oracle minimizes the number of state
relocations by monitoring the workload, and re-computing an optimized
partitioning on demand using a static partitioning algorithm. 
When a client submits a command, it first contacts the oracle to discover the partitions on
which state variables are stored. If the command accesses variables in multiple
partitions, the oracle issues a move command to the partitions, re-locating
variables to a single partition. When re-locating a variable, the oracle is
faced with a choice of which partition to use as a destination. \dynastar
chooses the partition for relocation (i.e., one that would minimize the number
of state relocations) by partitioning the workload graph using the
METIS~\cite{abou2006multilevel} partitioner. To track object location without
compromising scalability, in addition to information about the location of state
variables in the oracle, each client caches previous consults to the oracle. As
a result, the oracle is only contacted the first time a client accesses a
variable or after a variable changes its partition. Under the assumption of
locality, we expect that most queries to the oracle will be accurately resolved
by the client's cache.

\section{Thesis outline}
\label{sec:structure}

The rest of the thesis is organized as follows. Chapter 2 provides the
background for the thesis, by describing the system model and the communication
primitives used throughout this thesis. Chapter 3 considers the general problem
of scaling replicated systems. Chapter 4 discusses \dssmr, a dynamic
partitioning scheme for \smr, explaining how it may achieve dynamic state
reconfiguration while still maintaining strong consistency for \smr. Chapter 5
presents \dynastar, which combines the dynamic repartitioning state technique
and the centralized oracle to maintain a global view of the workload and partition
application state. Chapter 6 surveys related work on scaling state machine
replication. Finally, Chapter 7 concludes the thesis and proposes future
research directions.
