\chapter[Conclusion]{Conclusion}

With the explosive growth of the Internet, everyday online activities are
increasingly dependent on the performance and reliability of large-scale
distributed systems, such as e-banking, social networks, and e-commerce
platforms. Over the years, state machine replication has become the 
standard approach to providing highly available stateful services. By
replicating deterministic services across a number of servers, the replicated
service is available as long as the total number of failures does not exceed a
certain threshold. The strong consistency guarantee of state machine replication
hides the complexity of data replication among independent replicas, and makes the
underlying designs simpler. However, scaling such a replicated system while
preserving the strong consistency guarantee is not trivial, since each replica
must execute every command. To address this problem, several systems have
investigated the use of state partitioning in the context of SMR, allowing
client commands to be executed on a subset of replicas. 

In this dissertation, we have explored the space of highly available
and scalable systems from a performance perspective and proposed new solutions
to improve efficiency. The contribution of this thesis is centered on
scaling performance of a replicated state machine system. First, we presented a
general comparison of several approaches to scaling the performance of a
partitioned replicated system in the distributed system and database
communities, by using an abstract framework for coordination. Second, we
introduced \dssmrlong{} (\dssmr), a scalable variant of the well-known state
machine replication technique. \dssmr\ implements dynamic state partitioning to
adapt to different access patterns throughout the execution while scaling
throughput with the number of partitions and ensuring linearizability. Third, we
present \dynastar, a partitioning strategy for scalable state machine
replication. Different from \dssmr{}, \dynastar performs well in all workloads
evaluated. When the state can be perfectly partitioned, \dynastar converges more
quickly than DS-SMR; when partitioning cannot avoid cross-partition commands, it
largely outperforms DS-SMR. The key insights of \dynastar are to build a
workload graph on-the-fly and use an optimized partitioning of the workload
graph, computed with an online graph partitioner, to decide how to efficiently
move state variables. 

\section{Research assessment}

This dissertation presents three contributions: (i) a generic framework for
scaling partitioned replicated system, (ii) a dynamic partitioning scheme for SMR
(\dssmr), and (iii) an optimized partitioning scheme for SMR (\dynastar). In the following,
we review and discuss the most salient aspects of these contributions.
%some of the important lessons that we have learned during this research.

\paragraph{Survey on scalable partitioned replicated systems.}
Scaling a replicated system has been a topic of interest for many years. The
service is replicated on a number of servers, to tolerate a certain degree of
failure. Partitioning (or sharding) is a technique that divides the state of a
service in multiple partitions, so the load could be  equally distributed among
partitions. A partitioned replicated protocol can be described using five
generic phases: (i) the client submits a request to the system, (ii) the replicas
of the involved partitions coordinate with each other to synchronize the execution
of the request, (iii) the request is executed, (iv) the involved partitions
agree on the result of the execution, and (v) the system sends the outcome of the
request to the client. Some approaches may skip or simplify some phases: Google
Spanner \cite{corbett2013spanner} uses two-phase commit and two-phase locking
for concurrency control and does not need to order the transactions across
partitions; Calvin \cite{calvin}, on the other hand, orders the transactions in
a global log, deterministically executes transactions on all involved replicas,
and removes the overhead of coordinating after the execution phase.

\paragraph{Dynamic partitioning for state machine replication.} 
Classic SMR provides configurable fault tolerance and strong consistency but
limited performance scalability, since all replicated nodes must execute the same
sequence of commands. Some recent proposals have extended state machine
replication with sharding. Essentially, these approaches partition (shard) the
service state and replicate each partition. Commands that access state in a
single partition are handled as in classic state machine replication.\dssmr\ is
the technique that allows a partitioned SMR system to dynamically reconfigure
its data placement on-the-fly. \dssmr\ repartitions the service state
dynamically, based on the workload. When a command needs variables from
different partitions, those variables are first moved to the same partition.
Then, the command is executed as a single-partition command. To reduce the
negative effects of skewed load among partitions, the destination partition is chosen
randomly. As a result, variables that are usually accessed together will tend to
stay in the same partition, significantly improving scalability. We evaluate the
performance of \dssmr\ with a scalable social network application. The results
demonstrate that \dssmr\ could provide a scalable performance under workloads
that exhibit strong locality.

\paragraph{Optimized partitioning for state machine replication.}
Sharding and replication are the mechanisms of choice of most scalable and
fault-tolerant distributed systems. The performance of a partitioned system,
however, heavily depends on the partitioning of the data: in order to scale,
most commands must involve a single shard, and load must be balanced across
shards. Estimating a good partitioning of the application state is challenging
since it requires a priori information about the workload. Moreover, even if
such information is available, access patterns may change during system
execution. A good partitioning of the data for uniform access patterns may lead
to poor performance under skewed access patterns. \dynastar is a partitioning
strategy for scalable state machine replication. Differently from \dssmr{},
\dynastar performs well in workloads that exhibit strong and weak locality. In
the presence of strong locality, it converges more quickly than \dssmr{}; in the
presence of weak locality, it largely outperforms \dssmr{}. The key insights of
\dynastar are to build a workload graph on-the-fly and use an optimized
partitioning of the workload graph, computed with an online graph partitioner,
to decide how to efficiently move state variables. We describe \dynastar design
and implementation, and presents a detailed performance evaluation using two
benchmarks, a social network based on real data and TPC-C.

\section{Future directions}

The main objectives of this dissertation are to explore the space of techniques
for scalable, strongly consistent replicated systems. However, many questions
remain open, so we point here at possible research directions.

\paragraph{Remote updating objects.} \dynastar and \dssmr\ have similar execution
models: moving objects that are required by a command to the same partition and
executing the command against that partition. \dynastar has an extra step of
returning back updated objects to their original locations to preserve the
optimized partitioning. The operations of moving objects back and forth impose
significant overhead on the performance of the system. 
One direction to remove this overhead is to let partitions update the objects
remotely by making use of remote direct memory access (RDMA)
\cite{recio2007remote,dragojevic2014farm}. Essentially, RDMA is an approach
that allows a host to access the memory of another host without involving the
processor at the remote host. RDMA enables zero-copy transfers, low-latency
communication and reduces CPU overhead by bypassing the OS kernel and by
implementing several layers of the network stack in hardware. \dynastar powered
by RDMA could allow partitions to read or write an object
of a remote partition, without the need to relocate the object. However, 
handling the synchronization of the concurrent remote accesses is not trivial.
It is worthwhile to study synchronization techniques and propose new
and efficient protocols that can combine \dynastar with RDMA.

\paragraph{Optimal scalable atomic multicast protocol.} 
In order to provide scalable performance, \dssmr\ and \dynastar make use of
state partitioning. The requests from the client are only sent to the replicas
of the partitions that store the data being accessed. Traditional SMR relies on
total message ordering, but in the case of \dssmr\ and \dynastar, this may
become a bottleneck. To address this problem, \dssmr\ and \dynastar do not
totally order the requests but use a partial order multicast primitive instead,
which avoids enforcing a precedence relation between messages that do not share
the same destinations. This allows the ordering layer (i.e., the atomic
multicast protocol) to be itself also scalable.  As one of the directions for
expanding this study, a scalable atomic multicast protocol that
has optimal latency, while providing high throughput, would be the ideal
primitive for our system to build on.


\paragraph{Incremental partitioning.} Our default implementation of \dynastar
uses METIS \cite{Abou-Rjeili:2006} to provide a partitioning based on the
workload graph. Although METIS does not necessarily produce the best possible
partitioning of the workload graph, it offers a good compromise between
performance and partitioning quality. However, METIS needs to repartition graphs
from scratch every time a change is introduced in the graph or the number of
partitions. This could result in having \dynastar relocating the whole graph
\cite{SerafiniTEPAS16}. Techniques that provide incremental graph
partitioning are orthogonal to our paradigm but could be used to further
improve \dynastar performance.


\paragraph{Decentralized partitioning.} Even though defining optimized
partitioning by using a centralized component (i.e., the oracle) shows its
advantages over the decentralized approach, the drawback of this approach is the
scalability as the oracle is still prone to becoming a bottleneck in some cases.
This may happen if (i) the workload becomes too big to be stored in a single
machine, or (ii) the workload is very dynamic (i.e., the access patterns change 
frequently), which reduces the efficiency of the client cache and 
increases the queries to the oracle. As another direction for improving this
work, it is worthwhile to come up with a decentralized graph partitioning and
mapping that could help to remove the centralized component.
