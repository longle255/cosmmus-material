\section{Correctness criterion}
\label{sec:dynastar-correctcrit}

Our consistency criterion is linearizability.  A system is \emph{linearizable}
if there is a way to reorder the client commands in a sequence that (i)~respects
the semantics of the commands, as defined in their sequential specifications,
and (ii)~respects the real-time precedence of commands~\cite{Attiya04}.

\subsection{Correctness proof}
To prove that \dynastar ensures linearizability, we must show that for any
execution $\sigma$ of the system, there is a total order $\pi$ on client
commands that (i)~respects the semantics of the commands, as defined in their
sequential specifications, and (ii)~respects the real-time precedence of
commands~(\S\ref{sec:dynastar-correctcrit}).
%
Let $\pi$ be a total order of operations in $\sigma$ that respects $<$, the
order atomic multicast induces on commands.

To argue that $\pi$ respects the semantics of  commands, let $C_i$ be the $i$-th
command in $\pi$ and $p$ a process in partition $\ppm_p$ that executes $C_i$.
%Let $C_i$ be the $i$-th command in $\pi$ and $p$ a process in partition
%$\ppm_p$ that executes $C_i$.
We claim that when $p$ executes $C_i$, it has updated values of variables in
$vars(C_i)$, the variables accessed by $C_i$. We prove the claim by induction on
$i$. The base step trivially holds from the fact that variables are initialized
correctly. Let $v \in vars(C_i)$, $C_v$ be the last client command before $C_i$
in $\pi$ that accesses $v$, and $q$ a process in $\ppm_q$ that executes $C_v$.
From the inductive hypothesis, $q$ has an updated value of $v$ when it executes
$C_v$. There are two cases to consider: (a)~$p = q$. In this case, $p$ obviously
has an updated value of $v$ when it executes $C_i$ since no other command
accesses $v$ between $C_v$ and $C_i$. (b)~$p \neq q$. Since processes in the
same partition execute the same commands, it must be that $\ppm_p \neq \ppm_q$.
From the algorithm, when $q$ executes $C_v$, $v \in \ppm_q$ and when $p$
executes $C_i$, $v \in \ppm_p$. Thus, $q$ executed a command to move $v$ to
another partition after executing $C_v$ and $p$ executed a command to move $v$
to $\ppm_p$ before executing $C_i$. Since there is no command that accesses $v$
between $C_v$ and $C_i$ in $\pi$, $q$ has an updated $v$ when it executes $C_v$
(from inductive hypothesis), and $p$ receives the value of $v$ at $q$, it
follows that $p$ has an updated $v$ when it executes $C_i$.

% DISCLAIMER: in the proof below, I assumed a single process per partition.
% While I'm pretty sure it works for multiple processes per partition, in some
% future refinement this should be added to the proof. Also see related picture
% showing cases (a) and (b) below. Perhaps we should include such a picture too
% in a future version.
%
We now argue that there is a total order $\pi$ that respects the real-time
precedence of commands in $\sigma$. Assume $C_i$ ends before $C_j$ starts, or
more precisely, the time $C_i$ ends at a client is smaller than the time $C_j$
starts at a client, $\tec(C_i) < \tsc(C_j)$. Since the time $C_i$ ends at the
server from which the client receives the response for $C_i$ is smaller than the
time $C_i$ ends at the client, $\tes(C_i) < \tec(C_i)$, and the time $C_j$
starts at the client is smaller than the time $C_j$ starts at the first server,
$\tsc(C_j) < \tss(C_j)$, we conclude that $\tes(C_i) < \tss(C_j)$.

We must show that either $C_i < C_j$; or neither $C_i < C_j$ nor $C_j < C_i$.
For a contradiction, assume that $C_j <  C_i$ and let $C_j$ be executed by
partition $\ppm_j$.

There are two cases:
\begin{enumerate}
\item[(a)] $C_i$ is a client command executed by $\ppm_j$. In this case, since
$C_i$ only starts after $C_j$ at a server, it follows that $\tes(C_j) <
\tss(C_i)$, a contradiction.
\item[(b)] $C_i$ is a client command executed by $\ppm_i$ that first involves a
move of variables $vars$ from $\ppm_j$ to $\ppm_i$. At $\ppm_j$, $\tes(C_j) <
\tss(global(vars,\ppm_j,\ppm_i))$ since the move is only executed after $C_j$
ends. Since the move only finishes after variables in $vars$ are in $\ppm_i$ and
$C_i$ can be executed, it must be that
%\lle{move only finishes when $P_j$ gets back its variable}
$\tes(global(vars,\ppm_j,\ppm_i)) < \tss(C_i)$. We conclude that $\tes(C_j) <
\tss(C_i)$, a contradiction.
\end{enumerate}
Therefore, either $C_i < C_j$ and from the definition of $\pi$, $C_i$ precedes
$C_j$ or neither $C_i < C_j$ nor $C_j < C_i$, and there is a total order in
which $C_i$ precedes $C_j$.
%\begin{figure} \centering
%\includegraphics[width=1.2\linewidth,angle=-90]{figures/IMG_7203.JPG}
%\caption{Two cases in proof.} \end{figure}

For termination, we argue that every correct client eventually receives a
response for every command $C$ that it issues. This assumes that every partition
(including the oracle partition) is always operational, despite the failure of
some servers in the partition. For a contradiction, assume that some correct
client submits a command $C$ that is not executed. Atomic multicast ensures that
$C$ is delivered by the involved partition. Therefore, $C$ is delivered at a
partition that does not contain all the variables needed by $C$. As a
consequence, the client retries with the oracle, which moves all variables to a
single partition and requests the destination partition to execute $C$, a
contradiction that concludes our argument.


