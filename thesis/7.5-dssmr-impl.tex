\section{Implementation}
\label{sec:dssmr-implementation}

In this section, we describe \dssmrlibname{}, a library that implements both
\ssmr\ and \dssmr{}, and \dssmrappname{}, a scalable social network application
built with \dssmrlibname{}. \dssmrlibname\ and \dssmrappname\ were both
implemented in Java. \lle{Our \dssmr{} prototype uses the Ridge atomic multicast
library \cite{bezerra2015ridge}, available as open
source.\footnote{https://bitbucket.org/kdubezerra/ridge}. Ridge is a
Paxos-based atomic multicast protocol where each message is initially forwarded
to a single f distributor, whose responsibility is to propagate the message to
all other destination. Processes in Ridge alternate in the role of distributor
to utilize all bandwidth available in the system}
\fxnote{Adding details of multicast library}

\subsection{\dssmrlibname}

To implement a replicated service with \dssmrlibname{}, the developer (i.e., service
designer) must extend three classes: PRObject, StateMachine, OracleStateMachine.

\textbf{The PRObject class.} \dssmrlibname{} supports partial replication (i.e., some
objects may be replicated in some partitions, not all). Therefore, when
executing a command, a replica might not have local access to some of the
objects involved in the execution of the command. The developer informs
\dssmrlibname{} which object classes are partially replicated by extending the
PRObject class. Each object of such a class is stored either locally or remotely,
but the application code is agnostic to that. All calls to methods of such
objects are intercepted by \dssmrlibname{}, transparently to the developer.

% are you still using the aspect? do methods actually get intercepted by
% D-Eyrie?

\textbf{The StateMachine class.} This class implements the logic of the server
proxy. The application server class must extend the StateMachine class. To
execute commands, the developer must provide an implementation for the method
executeCommand(Command). The code for such a method is agnostic to the existence
of partitions. In other words, it can be exactly the same as the code used to
execute commands with classical state machine replication (i.e., full
replication). \dssmrlibname{} is responsible for handling all communication between
partitions and oracle transparently. To start the server, method
runStateMachine() is called. Method createObject() also needs to be implemented,
where the developer defines how new state objects are loaded or created.

\textbf{The OracleStateMachine class.} This class implements the logic of the
oracle proxy. It extends StateMachine, so the oracle can be deployed similarly
to a fault-tolerant partition in the original \ssmr{}. Class OracleStateMachine
has a default implementation, but the developer is encouraged to override its
methods. Method extractObject(Command) returns the set of objects accessed by
the command. It should be overridden by the application so that the client proxy
can relocate all necessary objects to a destination partition before executing
the application command.
% didn't understand this: ``in the case there are hidden objects in the command.''
Method getTargetPartition(Set$\langle$Object$\rangle$) returns a particular
partition to which objects should be moved, when they are not in the same
partition yet, in order to execute an application command that accesses those
objects. The default implementation of the method returns a random partition.
The developer can override it in order to further improve the distribution of
objects among partitions. For instance, the destination partition could be
chosen based on an attribute of the objects passed to the method.

The client proxy is implemented in class Client, which handles all communication
of the application client with the partitioned service. The client proxy
provides methods \emph{sendCreate(Command, Callback)}, \emph{sendAccess(Command,
Callback)}, and \emph{sendDelete(Command, Callback)}. The client proxy's default
behavior is to keep retrying commands (and fallback to \ssmr\ in case of too
many retries) and only call back the application client when the command has
been successfully executed. However, the developer can change this behavior by
overriding the error() method of Callback. The error() method is called when a
$retry$ reply is received.

% i don't understand what the application can actually change with the error()
% method

\subsection{\dssmrappname}

We implemented \dssmrappname{}, a social network application similar to Twitter,
using \dssmrlibname{}. Twitter is an online social networking service in which users
can post 140-character messages and read posted messages of other users. The API
consists basically of: post (user publishes a message), follow (user starts
following another user), unfollow (user stops following someone), and
getTimeline (user requests messages of all people whom the user follows).

State partitioning in \dssmrappname\ is based on users' interest. A function $f(uid)$
returns the partition that user with id $uid$ should belong to, based on the
user's interest. Function $f$ is implemented in method getObjectPlacement(User)
of class \dssmrappname{}Oracle, which extends OracleStateMachine (class User extends
PRObject). Taking into account that a typical user probably spends more time
reading messages (i.e., issuing getTimeline) than writing them (i.e., issuing
post), we decided to optimize getTimeline to be single-partition. This means
that, when a user requests his or her timeline, all messages should be available
in the partition that stores that user's data, in the form of a materialized
timeline (similarly to a materialized view in a database). To make this
possible, whenever a post request is executed, the message is inserted into the
materialized timeline of all users that follow the one that is posting. Also,
when a user starts following another user, the messages of the followed user are
inserted into the follower's materialized timeline as part of the command
execution; likewise, they are removed when a user stops following another user.
Because of this design decision, every getTimeline request accesses only one
partition, follow and unfollow requests access objects on at most two
partitions, and post requests access up to all partitions. The \dssmrappname{} client
does not need any knowledge about partitions, since it uses method
sendAccessCommand(command) of the \dssmr{} client proxy to issue its commands.

One detail about the post request is that it needs access to all users that
follow the user issuing the post.
%Thus the list of follower need to be attached in the post command. However, the
%(Chirper) client cannot know for sure who follows the user: it keeps a cache of
%followers, but such cache can become stale if a different user starts following
%the poster.
To ensure linearizability when executing a post request, the \dssmrappname\ server
overrides the extractObject(command) method to check if all followers that will
be accessed by the command are available in the local partition (i.e., the
partition of the server executing the post command). If this is the case, the
request is executed. Otherwise, the server sends a $retry(\gamma)$ message,
where $\gamma$ is the complete set of followers of the user who was posting.
Then, the \dssmrappname\ server proceeds to the next command. Upon receiving the
$retry(\gamma)$ message, the client proxy tries to move all users in $\gamma$ to
the same partition before retrying to execute the post command.

