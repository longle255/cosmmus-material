\begin{abstract}
  Today's online services must meet strict availability and performance
  requirements. State machine replication (SMR), one of the most fundamental
  approaches to increasing the availability of services without sacrificing
  strong consistency, provides configurable availability but limited performance
  scalability. The lacking of scalability of SMR due to the fact that every
  replica has to execute all commands, which limits the overall performance by the
  throughput of a single replica, so adding servers does not increase the
  maximum throughput. Scalable State Machine Replication (S-SMR) achieves
  scalable performance by partitioning the service state and coordinating the
  ordering and execution of commands. While S-SMR scales the performance of
  single-partition commands with the number of deployed partitions, replica
  coordination needed by multi-partition commands introduces an overhead in the
  execution of multi-partition commands. In this thesis, we propose and
  implement the following ideas: (i) Dynamic scalable state machine replication
  (DS-SMR), (ii) Dynastar: Optimized partitioning for SMR. DS-SMR addresses the
  problem of S-SMR by allowing repartitioning the service state dynamically,
  based on the workload. Variables that are usually accessed together are moved
  to the same partition, which significantly improves scalability. To provide
  better partitioning for DS-SMR, we develop \dynastar, a novel approach to
  scaling SMR. \dynastar also uses dynamically repartitioning state technique,
  combined with a centralized oracle, to maintain a global view of workload and
  inform heuristics about data placement. Using this oracle, \dynastar is able
  to adapt to workload changes over time, while also minimizing the number of
  state changes. The performance evaluation using two benchmarks, a social
  network based on real data and TPC-C, shows that \dynastar is a practical
  technique that achieves excellent throughput.
\end{abstract}
