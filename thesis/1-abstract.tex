\begin{abstract}
  State machine replication (SMR) is a well-known technique that guarantees
  strong consistency (i.e., linearizability) to online services. However, SMR
  lacks scalability since every replica executes all commands, which limits the
  overall performance by the throughput of a single replica, so adding servers
  does not increase the maximum throughput. Scalable SMR (\ssmr) addresses this
  problem by partitioning the service state, allowing commands to execute only
  in some replicas and providing scalability while still ensuring
  linearizability. One problem is that \ssmr only works with static partitioning
  scheme, and quickly saturates when executing multi-partition commands, as
  partitions must communicate. In this research, we propose \dynastar, a novel
  approach to scaling SMR. \dynastar uses dynamically repartitioning state
  technique, combined with a centralized oracle to maintain a global view of
  workload and inform heuristics about data placement. Using this oracle,
  \dynastar is able to adapt to workload changes over time, while also
  minimizing the number of state changes. The preliminary results show that
  \dynastar is a practical technique that achieves excellent throughput.
\end{abstract}
