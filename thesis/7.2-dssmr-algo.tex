\section{Detailed algorithm}
\label{sec:dssmr-detail}


When issuing a command, the application simply forwards the command to the
client proxy and waits for the reply. Consulting the oracle and multicasting the
command to different partitions is done internally by the proxy at the client.
Algorithms~\ref{alg:client_proxy}, \ref{alg:server_proxy}, and
\ref{alg:oracle_proxy} describe in detail how the \dssmr\ proxy works
respectively at client, server and oracle processes. Every server proxy at a
server in $\ssm_i$ has only partial knowledge of the partitioning: it knows only
which variables belong to $\ppm_i$. The oracle proxy has knowledge of every
$\ppm \in \Psi$. To maintain such a global knowledge, the oracle must \amdel{}
every command that creates, moves, or deletes variables. (In
Section~\ref{sec:dssmr-optm}, we introduce a caching mechanism to prevent the oracle
from becoming a performance bottleneck.)

\begin{algorithm}[t!]
\small

\begin{distribalgo}[1]

\vspace{1.0mm}

\INDENT{To issue a command $C$, the client proxy does:}

\vspace{1.0mm}

    \INDENT{\textbf{do}}
        \STATE \amcast$($oracle, $consult(C))$
        \STATE wait for $prophecy$
        \IF{$prophecy \in \{ok, nok\}$}
            \STATE $reply \leftarrow prophecy$
        \ELSE
            \STATE $C.dests \leftarrow \{\ppm : \exists \langle v, \ppm \rangle \in prophecy \}$
            \IF{$C$ is an $access(\omega)$ command and $|C.dests| > 1$}
                \STATE let $\ppm_d$ be one of the partitions in $C.dests$
                \label{algline:client:partition}
                \FOR{each $v \in \omega$}
                    \STATE // \textit{move $v$ to partition $\ppm_d$}
                    \STATE let $\ppm_s$ be $\ppm : \langle v, \ppm \rangle \in prophecy$
                    \IF{$\ppm_s \neq \ppm_d$}
                        \STATE $C_{move} \leftarrow move(v,\ppm_s,\ppm_d)$
                        \STATE $C_{move}.dests \leftarrow \{$oracle$,\ppm_s,\ppm_d\}$    
                        \STATE \amcast$(C_{move}.dests$, $C_{move})$
                    \ENDIF
                \ENDFOR
                \STATE $C.dests \leftarrow \{ \ppm_d \}$
            \ENDIF
            \IF{$C$ is $create$ or $delete$}
                \STATE $C.dests \leftarrow dests \cup \{oracle\}$
            \ENDIF
            \STATE \amcast$(C.dests$, $C)$
            \STATE wait for $reply$
        \ENDIF
    \ENDINDENT
    \STATE{\textbf{while} $reply = retry$ // \textit{after many retries, fall back to \ssmr}}
    \STATE return $reply$ to the application client
\ENDINDENT

\caption{\dssmr\ Client Proxy}
\label{alg:client_proxy}
\end{distribalgo}
\end{algorithm}
\fxnote{Clarifying algorithm}
\textbf{The client proxy.} To execute a command $C$, the proxy first consults
the oracle. The oracle knows all state variables and which partition contains
each of them. Because of this, the oracle may already tell the client whether
the command can be executed or not. Such is the case of the $access(\omega)$
command: if there is a variable $v \in \omega$ that the command tries to read or
write and $v$ does not exist, the oracle already tells the client that the
command cannot be executed, by sending $nok$ as the prophecy. A $nok$ prophecy
is also returned for a $create(v)$ command when $v$ already exists. For a
$delete(v)$ command when $v$ already does not exist, an $ok$ prophecy is
returned. If the command can be executed, the client proxy receives a prophecy
containing a pair $\langle v, \ppm \rangle$, for every variable $v$ created,
accessed or deleted by the command. \lle{If the prophecy regarding an
$access(\omega)$ command contains multiple partitions, the client proxy randomly chooses
one of them,} $\ppm_d$, and tries to move all variables in $\omega$ to $\ppm_d$.
Then, the command $C$ itself is multicast to $\ppm_d$. As discussed in
Section~\ref{sec:dssmr-idea}, there is no guarantee that an interleave of
commands will not happen, even if the client waits for the replies to the move
commands. For this reason, and to save time, the client proxy multicasts all
move commands at once. Commands that change the partitioning (i.e., create and
delete) are also multicast to the oracle. If the reply received to the command
is $retry$, the procedure restarts: the proxy consults the oracle again,
possibly moves variables across partitions, and multicasts $C$ to the
appropriate partitions once more. After reaching a given threshold of retries
for $C$, the proxy falls back to \ssmr{}, multicasting $C$ to all partitions
(and the oracle, in case $C$ is a create or delete command), which ensures the
command's termination.

\begin{algorithm}[t!]
\small

\begin{distribalgo}[1]

\vspace{1.0mm}

\INDENT{To execute a command $C$, the server proxy in partition $\ppm$ does:}

    \vspace{1.0mm}
    
    \INDENT{\textbf{when} \rmdel$( \langle val, C \rangle )$}
        \STATE $rcvd\_msgs \leftarrow rcvd\_msgs \cup \{\langle val, C \rangle\}$
    \ENDINDENT

    \vspace{1.0mm}

    \INDENT{\textbf{when} \amdel$(C)$}

    \vspace{1.0mm}

        \IF{\underline{$C$ is an $access(\omega)$ command}}
            \IF{$\exists v \in \omega : v \not\in \ppm$}
                \STATE reply with $retry$
            \ELSE
                \STATE have the command executed by the application server
                \STATE send the reply to the client
            \ENDIF
        

        \vspace{1.0mm}
    
        \ELSIF{\underline{$C$ is a $move(v,\ppm_s,\ppm_d)$ command}}
            \IF{$\ppm = \ppm_s$}
                \IF{$v \in \ppm$}
                    \STATE \rmcast$(\ppm_d$,$\langle v, C \rangle)$
                    \STATE $\ppm \leftarrow \ppm \setminus \{v\}$
                \ELSE
                    \STATE \rmcast$(\ppm_d$,$\langle null, C \rangle)$
                \ENDIF
            \ELSE
                \STATE wait until $\exists val : \langle val, C \rangle \in rcvd\_msgs$
                \IF{$val \neq null$}
                    \STATE $v \leftarrow val$
                    \STATE $\ppm \leftarrow \ppm \cup \{v\}$
                \ENDIF
            \ENDIF
        
        \vspace{1.0mm}
    
        \ELSIF{\underline{$C$ is a $create(v)$ command}}
            \STATE wait until $\langle val, C \rangle \in rcvd\_msgs$
            \IF{$val = ok$}
                \STATE $\ppm \leftarrow \ppm \cup \{v\}$
%                \STATE reply with $success$
%            \ELSE
%                \STATE reply with $retry$
            \ENDIF
        
        \vspace{1.0mm}
        
        \ELSIF{\underline{$C$ is a $delete(v)$ command}}
            \IF{$v \in \ppm$}
                \STATE $\ppm \leftarrow \ppm \setminus \{v\}$
%                \STATE reply with $success$
%            \ELSE
%                \STATE reply with $retry$
            \ENDIF
        \ENDIF
    \ENDINDENT
\ENDINDENT

\caption{\dssmr\ Server Proxy}
\label{alg:server_proxy}
\end{distribalgo}
\end{algorithm}

\textbf{The server proxy.} Upon delivery, access commands are intercepted by the
\dssmr\ proxy before they are executed by the application server. In \dssmr{},
every access command is executed in a single partition. If a server proxy in
partition $\ppm$ intercepts an $access(\omega)$ command that accesses a variable
$v \in \omega$ that does not belong to $\ppm$, it means that the variable is in
some other partition, or it does not exist. Either way, the client should retry
with a different set of partitions, if the variable does exist. To execute a
$delete(v)$ command, the server proxy at partition $\ppm$ simply removes $v$
from partition $\ppm$, in case $v \in \ppm$. In case $v \not\in \ppm$, it might
be that the variable exists but belongs to some other partition $\ppm'$. Since
only the oracle and the servers at $\ppm'$ have this knowledge, it is the oracle
who replies to delete commands.

The \dssmr\ server and oracle proxies coordinate to execute commands that create or
move variables. Such coordination is done by means of \rmcast{}. When a
$create(v)$ command is delivered at $\ppm$, the server proxy waits for a message
from the oracle, telling whether the variable can be created or not, to be
\rmdel{}ed. Such a message from the oracle is necessary because $v$ might not
belong to $\ppm$, but it might belong to some other partition $\ppm'$ that
servers of $\ppm$ have no knowledge of. If the create command can be executed,
the oracle can already reply to the client with a positive acknowledgement,
saving time. This can be done because atomic multicast guarantees that all
non-faulty servers at $\ppm$ will eventually deliver and execute the command. As
for move commands, each $move(v,\ppm_s,\ppm_d)$ command consists of moving
variable $v$ from a source partition $\ppm_s$ to a destination partition
$\ppm_d$. If the server's partition $\ppm$ is the source partition (i.e., $\ppm$
= $\ppm_s$), the server proxy checks whether $v$ belongs to $\ppm$. If $v \in
\ppm$, the proxy \rmcast{}s $\langle v, C \rangle$ to $\ppm_d$, so that servers
at the destination partition know the most recent value of $v$; $C$ is sent
along with $v$ to inform which move command that message is related to. If $v
\not\in \ppm$, a $\langle null, C \rangle$ message is \rmcast\ to $\ppm_d$,
informing $\ppm_d$ that the move command cannot be executed.

\begin{algorithm}[t!]
\small

\begin{distribalgo}[1]

\vspace{1.5mm}

\INDENT{To execute a command $C$, the oracle proxy does:}

    \vspace{1.5mm}

    \INDENT{\textbf{when} \amdel$(C)$}

    \vspace{1.5mm}

        \IF{\underline{$C$ is a $consult(C_c)$ command}}
            \STATE $prophecy \leftarrow \emptyset$

            \IF{$C_c$ is an $access(\omega)$ command}
                \IF{$\exists v \in \omega : (\nexists \ppm \in \Psi : v \in \ppm)$}
                    \STATE $prophecy \leftarrow nok$
                \ELSE
                    \FOR{each $v \in \omega$}
                        \STATE $prophecy \leftarrow prophecy \cup \{\langle v, \ppm \rangle \}:v \in \ppm$
                    \ENDFOR
                \ENDIF

            \ELSIF{$C_c$ is a $create(v)$ command}
                \IF{$\exists \ppm \in \Psi : v \in \ppm$}
                    \STATE $prophecy \leftarrow nok$
                \ELSE
                    \STATE $\ppm \leftarrow$ initial partition, defined by application rules
                    \STATE $prophecy \leftarrow \{\langle v, \ppm \rangle \}$
                \ENDIF

            \ELSIF{$C_c$ is a $delete(v)$ command}
                \IF{$\nexists \ppm \in \Psi : v \in \ppm$}
                    \STATE $prophecy \leftarrow ok$
                \ELSE
                    \STATE $prophecy \leftarrow \{\langle v, \ppm \rangle \} : v \in \ppm$
                \ENDIF
                
            \ENDIF
            
            \STATE send $prophecy$ to the client

        \vspace{1.5mm}

        \ELSIF{\underline{$C$ is a $move(v,\ppm_s,\ppm_d)$ command}}
            \IF{$v \in \ppm_s$}
                \STATE $\ppm_s \leftarrow \ppm_s \setminus \{v\}$
                \STATE $\ppm_d \leftarrow \ppm_d \cup      \{v\}$
            \ENDIF

        \vspace{1.0mm}
    
        \ELSIF{\underline{$C$ is a $create(v)$ command}}
            \STATE let $\ppm_c$ be $\ppm : \{\ppm\} = C.dests \setminus \{$oracle$\}$
            \IF{$\nexists \ppm \in \Psi : v \in \ppm$}
                \STATE $outcome \leftarrow ok$
            \ELSE
                \STATE $outcome \leftarrow nok$
            \ENDIF
            \STATE \rmcast$(\ppm_c, \langle outcome, C \rangle )$
            \STATE send $outcome$ to the client
        
        \vspace{1.0mm}
        
        \ELSIF{\underline{$C$ is a $delete(v)$ command}}
            \STATE let $\ppm_d$ be $\ppm : \{\ppm\} = C.dests \setminus \{$oracle$\}$
            \IF{$\nexists \ppm \in \Psi: v \in \ppm$ or $v \in \ppm_d$}
                \STATE send $ok$    to the client
            \ELSE
                \STATE send $retry$ to the client
            \ENDIF
            
            
        \ENDIF
    \ENDINDENT
\ENDINDENT

\caption{\dssmr\ Oracle Proxy}
\label{alg:oracle_proxy}
\end{distribalgo}
\end{algorithm}

\textbf{The oracle proxy.} One of the purposes of the oracle proxy is to make
prophecies regarding the location of state variables. Such prophecies are used
by client proxies to multicast commands to the right partitions. A prophecy
regarding an $access(\omega)$ command contains, for each $v \in \omega$, a pair
$\langle v, \ppm \rangle$, meaning that $v \in \ppm$. If any of the variables in
$\omega$ does not exist, the prophecy already tells the client that the command
cannot be executed (with a $nok$ value). For a $create(v)$ command, the prophecy
tells where $v$ should be created, based on rules defined by the application, if
$v$ does not exist. If $v$ already exists, the prophecy will contain $nok$, so
that the client knows that the create command cannot be executed. The prophecy
regarding a $delete(v)$ command has the partition that contains $v$, or
$ok$, in case $v$ was already deleted or never~existed.

Besides dispensing prophecies, the oracle is responsible for executing create,
move, and delete commands, coordinating with server proxies when necessary, and
replying directly to clients in some cases. For each $move(v,\ppm_s,\ppm_d)$
command, the oracle checks whether $v$ in fact belongs to the source partition
$\ppm_s$. If that is the case, the command is executed, moving $v$ to $\ppm_d$.
Each $create(v)$ command is multicast to the oracle and to a partition $\ppm$.
If $v$ already exists, the oracle tells $\ppm$ that the command cannot be
executed, by \rmcast{}ing $nok$ to $\ppm$. The oracle also sends $nok$ to the
client as reply, meaning that $v$ already exists. If $v$ does not exist, the
oracle tells $\ppm$ that the command can be executed, by \rmcast{}ing $ok$ to
$\ppm$. It also tells the client that the command succeeded with an $ok$ reply.
Finally, each $delete(v)$ command is multicast to the oracle and to a partition
$\ppm$, where the client proxy assumed $v$ to be located. If $v$ belongs to
$\ppm$, or $v$ does not exist, the oracle tells the client that the delete
command succeeded. Otherwise, that is, if $v$ exists, but $delete(v)$ was
multicast to the wrong partition, the oracle tells the client to retry.


