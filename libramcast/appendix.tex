%!TEX root =  main.tex

\newcommand{\Pend}{\ensuremath{\mathit{ToOrder}}\xspace}
\newcommand{\Done}{\ensuremath{\mathit{Ordered}}\xspace}
\newcommand{\Decided}{\ensuremath{\mathit{Decided}}\xspace}
\newcommand{\Buffer}{\ensuremath{\mathcal{B}}\xspace}

\clearpage
\section{Proof of correctness}

%\setcounter{proposition}{2}
\begin{proposition}
\textit{(Uniform integrity)~For any message $m$, every process $p$ delivers $m$ at most once, and only if $p$ is a destination of $m$ and $m$ was previously multicast.}
\end{proposition}
\vspace{2mm}
\noindent
{\sc Proof:} 
Process $p$ delivers $m$ at Task 6 if $m$'s state is \ordered. 
After delivering $m$, $p$ sets its state to \done, and thus $m$ cannot be delivered more than once.

Let $c$ be the client that multicasts $m$ to groups in $dst$, and let $p$ be in group $g$. 
From Task 6, $p$ only delivers $m$ if it is in $p$'s $M$ buffer and $m$'s state is \ordered. 
Message $m$'s state is set to \ordered\ in Task 5 if its current state is \mcast.
A message's state is set to \mcast\ in procedure $Relay$, which is invoked in two cases:
(a)~by client $c$ upon multicasting $m$ (Task 1) to groups in $dst$, in which case $g \in dst$; or 
(b)~by some process $q$ that suspects $c$ (Task 10), has $m$ in its buffer in state \mcast, and $g$ is a destination of $m$.
In case (b), $m$ was written in $q$'s buffer either (b.1) directly by $c$ or (b.2) indirectly by some other process.
In any case, there is some process $r$ such that $m$ is included in $r$'s buffer by $c$.
It follows from Task 1 that $p$ is a destination of $m$ and $m$ was multicast by client $c$.\hfill$\Box$

%\begin{lemma}
%If a correct process $p$ in $g$ includes tuple $T$ in \Pend, then eventually processes in $g$ decide on a sequence of tuples that contains $T$.
%\label{lemma:Y}
%\end{lemma}
%\noindent
%{\sc Proof:} 
%Process $p$ includes $T$ in \Pend either in Task 1 or in Task 2.
%In both cases, $T$ was r-delivered by $p$ and from the properties of reliable broadcast, every correct process in $g$ will r-deliver $T$ and include it in \Pend.
%Let $t$ be a time after which all faulty processes have failed.
%Thus, after $t$ there is a time when all \Pend sequences from processes that propose in consensus contain $T$.
%By the uniform integrity property of consensus, $T$ is eventually included in a decision of consensus.\hfill$\Box$
%
%
%\begin{lemma}
%For each correct process $p$ that has tuple $(\textsc{sync-hard}, h, x, m)$ in \Buffer, $p$ eventually replaces the entry by $(\textsc{final}, \bot, ts, m)$ in \Buffer where $ts$ is the maximum timestamp $x$ in the \textsc{sync-hard} tuples that concern $m$.
%\label{lemma:X}
%\end{lemma}
%\noindent
%{\sc Proof:} 
%To include $(\textsc{sync-hard}, h, x, m)$ in \Buffer, $p$ has decided on a sequence that contains either \\
%(a)~a $(\textsc{set-hard}, h, x, m)$ tuple if $m$ is local, or 
%(b)~a $(\textsc{sync-hard}, h, x, m)$ tuple if $m$ is global.
%In case (a), $(\textsc{sync-hard}, h, x, m)$ will be trivially replaced by $(\textsc{final}, \bot, x, m)$ in Task 5.
%In case (b), some process proposed a $\Pend \setminus \Done$ sequence that contains $(\textsc{sync-hard}, h, x, m)$.
%The \textsc{sync-hard} tuple is included in \Pend in Task 2 upon r-delivering tuple $(\textsc{send-hard}, h, x, m)$, which was r-multicast in Task 4, upon the decision of a sequence with $(\textsc{set-hard}, h, x, m)$.
%Thus, $(\textsc{set-hard}, h, x, m)$ was included in \Pend at Task 1, as a result of the r-delivery of $(\textsc{start}, \bot, \bot, m)$, which is r-multicast to all of $m$'s destinations.
%Every group $h$ in $m.dst$ upon r-delivering $(\textsc{start}, \bot, \bot, m)$ adds tuple $(\textsc{set-hard}, h, x, m)$ to \Pend, which from Lemma~\ref{lemma:Y} is eventually included in a consensus decision and results in the r-multicast of $(\textsc{send-hard}, h, x, m)$ to members of $m.dst$.
%When a process r-delivers $(\textsc{send-hard}, h, x, m)$, it adds $(\textsc{sync-hard}, h, x, m)$ to \Pend and, from Lemma~\ref{lemma:Y}, the tuple is decided in an instance of consensus, leading to the inclusion of $(\textsc{sync-hard}, h, x, m)$ in \Buffer.
%Once there is a tuple $(\textsc{sync-hard}, h, x, m)$ in \Buffer for each group $h$ in $m.dst$, $p$ replaces the \textsc{sync-hard} tuples by $(\textsc{final}, \bot, ts, m)$.\hfill$\Box$
%
%\begin{lemma}
%If a correct process $p$ includes $(\textsc{final}, \bot, ts, m)$ in \Buffer, then $p$ eventually a-delivers $m$.
%\label{lemma:Z}
%\end{lemma}
%\noindent
%{\sc Proof sketch:} 
%Assume for a contradiction that $q$ does not a-deliver $m$.
%Thus, there is some tuple $(z,h,y,m')$ in \Buffer such that $m \neq m'$ and $y < ts $.
%We first show that eventually any entry $(z,h,y,m')$ added in \Buffer after $(\textsc{final}, \bot, ts, m)$ is in \Buffer has a timestamp bigger than $ts$.
%Message $m$ only has a \textsc{final} tuple in \Buffer after it received \textsc{sync-hard} tuples from all of $m$'s destinations.
%When $q$ includes $(\textsc{sync-hard}, h, x, m)$ in \Buffer in Task 4, $q$ updates $C_h$ such that it contains the maximum between it current value and $x$.
%Since the next \textsc{set-hard} event that $q$ handles for a message $m''$ will increment $C_h$, it follows that $m''$ will have a timestamp bigger than $ts$.
%
%We now show that every message that contains a timestamp smaller than $m$'s final timestamp $ts$ is eventually a-delivered and removed from \Buffer.
%Let $(z,h,y,m')$ be an entry in \Buffer such that $y<ts$.
%Either $z$ is \textsc{final} or it is \textsc{sync-hard} and from Lemma~\ref{lemma:X} $z$ the tuple will eventually be replaced by a \textsc{final} tuple.
%Thus, from Task 5 message $m'$ will be eventually a-delivered and removed from \Buffer, a contradiction.
%We conclude then that $q$ eventually a-delivers $m$.\hfill$\Box$

\begin{lemma}
If a correct process $p$ has a message $m$ in its $M$ buffer in the \ordered\ state, then $p$ eventually delivers $m$.
\label{lemma:Z}
\end{lemma}
\noindent
{\sc Proof:} 
Assume for a contradiction that $q$ does not deliver $m$.
Thus, there is some message $m'$ in the buffer such that $m \neq m'$ and $m'$'s timestamp is smaller than $m$'s and $m'$ is not \ordered.
We first show that eventually any message added in the buffer after $m$ becomes \ordered\ has a timestamp bigger than $m$'s timestamp.
Message $m$ only becomes ordered after it has timestamps from all of $m$'s destinations.
When $q$ reads a timestamp $x$ for $m$ from some group, $q$ updates its clock such that it contains the maximum between it current value and $x$.
Since the next event that $q$ handles for a message $m''$ will increment its clock, it follows that $m''$ will have a timestamp bigger than $x$.

We now show that every message that contains a timestamp smaller than $m$'s final timestamp $ts$ is eventually delivered and its state set to \done.
To see why, let $m'$ be the message with the smallest timestamp in the buffer.
Thus, such a message is eventually delivered and its state set to \ordered.
Eventually, $m$ will be the message in the buffer with smallest timestamp and therefore delivered, a contradiction.
We conclude then that $q$ eventually delivers $m$.\hfill$\Box$


\vspace{2mm}
\setcounter{proposition}{1}
\begin{proposition}
\textit{(Validity) If a correct client $c$ multicasts a message $m$, then eventually all correct processes $p$ in $m$'s destination $dst$ deliver $m$.}
\end{proposition}
\noindent
{\sc Proof:} 
Upon multicasting $m$, $c$ relays $m$ to groups in $dst$ (see Task 1).
The Relay procedure then copies $m$ to the $M$ buffer of every correct process $q$ in groups in $dst$ and sets its state to \mcast.

We claim that for each $h \in dst$, $q$ will have a timestamp for $h$ that is acknowledged by a quorum of processes in $h$.
By the leader election oracle and the fact that each group has a majority of correct processes, group $h$ eventually has a stable correct leader $l$.
Either $l$ (a) executes Task 2 and proposes its clock value as $h$'s timestamp or (b) executes Task 7 to replace a suspected leader.
In (b), $l$ sends a \textsc{catch\_up} message to all processes and will receive for each group $g \in dst$ the timestamp proposed in $g$, if any, and the corresponding acknowledgements from processes in $g$.
For the case where $h=g$, $l$ will pick the timestamp decided by a previous leader or choose one if no timestamp has been decided (Task 9).
Thus, in both cases (a) and (b), the leader writes the chosen timestamp in the $M$ buffer of each process in $h$ and in the leaders of other groups in $dst$.
From Task 3, every follower in $h$ will acknowledge this timestamp in the buffer of each process in the destination of $m$.
From Task 4, when $l$ has a timestamp from $g \neq h$, $l$ writes the timestamp in the buffer of its followers, which concludes the claim.

Therefore, eventually $q$ has a timestamp for every group in $dst$, can compute $m$'s final timestamp, and set $m$'s state as \ordered.
From Lemma~\ref{lemma:Z}, $q$ eventually delivers $m$.\hfill$\Box$


\setcounter{proposition}{4}
\begin{proposition}
\textit{Uniform agreement}:~If a process $p$ delivers a message $m$, then eventually all correct processes $q$ in $m$'s destination $dst$ deliver $m$.
\end{proposition}
\noindent
{\sc Proof sketch:} 
For process $p$ to deliver $m$, from Task 6, $p$ has a timestamp for every group $h$ in $dst$ in the $M$ buffer such that $ts$ is the largest among these timestamps.
Moreover, there is no message $m'$ in the buffer $m \neq m'$, such that $ts < y$, where $y$ is a timestamp assigned to $m'$.
We consider two cases.

Case (a): $p$ and $q$ are in the same group $g$.
Process $p$ adds tuples to \Buffer in Task 4 only, and every tuple added to \Buffer, which can be of type \textsc{set-hard} and \textsc{sync-hard}, is created from an entry in \Decided and variable $C_h$.
Moreover, $C_h$ is modified only in Task 4, based on tuples in \Decided.
Since $p$ and $q$ decide on the same sequence of consensus values, it follows that unless $q$ fails, it executes the same sequence of steps as $p$ and eventually a-delivers $m$.

Case (b): $p$ and $q$ are in different groups.
To include $(\textsc{sync-hard}, h, x, m)$ in \Buffer, $p$ has decided on a sequence that includes the tuple, for each group $h \in m.dst$.
From consensus, some process $r$ proposed a sequence that contains $(\textsc{sync-hard}, h, x, m)$ in \Pend (Task 3).
So, $r$ r-delivered tuple $(\textsc{send-hard}, h, x, m)$, which was r-multicast to all of $m$'s destinations in Task 4.
As a consequence, processes in $q$'s group r-deliver message $(\textsc{send-hard}, h, x, m)$ and decide on a sequence that includes $(\textsc{sync-hard}, h, x, m)$.
We conclude that $r$ adds $(\textsc{sync-hard}, h, x, m)$ to \Buffer, for each of $m$'s destinations $h$.
From Lemma~\ref{lemma:Z}, $q$ eventually a-delivers $m$.\hfill$\Box$


\begin{proposition}
\textit{Uniform prefix order}:~For any two messages $m$ and $m'$ and any two processes $p$ and $q$ such that $\lbrace p, q\rbrace \subseteq m.\mathit{dst} \cap m'.\mathit{dst}$, if $p$ a-delivers $m$ and $q$ a-delivers $m'$, then either $p$ a-delivers $m'$ before $m$ or $q$ a-delivers $m$ before $m'$.
\end{proposition}
\noindent
{\sc Proof:} 
The proposition trivially holds if $p$ and $q$ are in the same group, so assume $p$ is in group $g$ and $q$ is in group $h$ and suppose, by way of contradiction, that $p$ does not a-deliver $m'$ before $m$ nor does $q$ a-deliver $m$ before $m'$. 
Without loss of generality, suppose that $m.ts < m'.ts$. 

We claim that $q$ inserts $m$ into \Buffer before a-delivering $m'$. 
In order for $m$ (respectively, $m'$) to be a-delivered by $p$ (resp., $q$), $p$'s (resp., $q$'s) \Buffer must contain tuples $(\textsc{sync-hard}, g, x, m)$ and $(\textsc{sync-hard}, h, y, m)$ (resp., $(\textsc{sync-hard}, g, x', m')$ and $(\textsc{sync-hard}, h, y', m')$).
From Task 4, for $p$ to include a \textsc{sync-hard} tuple in \Buffer, $p$ must have decided a sequence that contains $(\textsc{sync-hard}, g, x, m)$ (recall that $m$ is a global message).
%either $(\textsc{set-hard}, g, x, m)$ and $m$ is local or $(\textsc{sync-hard}, g, x, m)$ and m is global.
Thus, some process in $g$ included $(\textsc{sync-hard}, g, x, m)$ in \Pend, after r-delivering tuple  $(\textsc{send-hard}, g, x, m)$.
With a similar argument, some process in $g$ included $(\textsc{sync-hard}, g, x', m')$ in \Pend, after r-delivering tuple $(\textsc{send-hard}, g, x', m')$.
Let $r$ and $s$ be the processes that r-multicast messages $(\textsc{send-hard}, g, x, m)$ and $(\textsc{send-hard}, g, x', m')$, respectively, at Task 4.
Therefore, $r$ and $s$ decided sequences that include the \textsc{set-hard} tuples.
Assume that $(\textsc{set-hard}, g, x, m)$ is decided before $(\textsc{set-hard}, g, x', m')$.
Therefore, before r-multicasting $(\textsc{send-hard}, g, x', m')$, $s$ r-multicast $(\textsc{send-hard}, g, x, m)$.
From the FIFO properties of reliable multicast, $q$ r-delivered the tuples in the order above and we can show that $(\textsc{sync-hard}, g, x, m)$ appears in \Buffer before $(\textsc{set-hard}, g, x', m')$, which proves our claim.

Consequently, from the claim, $q$ a-delivers $m$ before $m'$ since $m.ts_q < m'.ts_q$, a contradiction that concludes the proof.
\hfill$\Box$

\begin{proposition}
\textit{Uniform acyclic order.
Let relation $<$ be defined such that $m < m'$ iff there exists a process that delivers $m$ before $m'$.
The relation $<$ is acyclic.}
\end{proposition}
\noindent
{\sc Proof sketch:} 
Suppose, by way of contradiction, that there exist messages $m_1, ..., m_k$ such that $m_1 < m_2 < ... < m_k < m_1$. 
From Task 5, processes a-deliver messages following the order of their final timestamps.
Thus, there must be processes $p$ and $q$ such that the final timestamps they assign to $m_1$, $m_1.ts_p$ and  $m_1.ts_q$, satisfy $m_1.ts_p < m_1.ts_q$, a contradiction since both $p$ and $q$ receive the same \textsc{sync-hard} tuples used to calculate $m_1$'s final timestamp in Task 5.


\hfill$\Box$

\begin{theorem}
RamCast implements atomic multicast.
\end{theorem}
\noindent
{\sc Proof:} 
This follows directly from Propositions 3 through 7.\hfill$\Box$

