%!TEX root =  main.tex
\section{Implementation}
\label{sec:implementation}

%This section presents implementation details and introduces competitor protocols.
%
%\subsubsection*{\libname}

We implemented a prototype of \libname in Java using jVerbs,\footnote{DiSNI
library \url{https://github.com/zrlio/disni}} an
open-source user-level networking library developed by IBM that supports RDMA
communication~\cite{stuedi2013jverbs}. jVerbs offers low latencies to applications running inside a Java
Virtual Machine by exposing RDMA network hardware resources directly to the JVM.
The source code of \libname is publicly available.\footnote{Link not yet available due to double blind review.}
%The source code of \libname is publicly available.\footnote{\url{https://github.com/longle255/libRamcastV3}}

In \libname, we applied a number of optimizations to further decrease latency
and improve performance. When establishing the connections between hosts, we
use two-sided operations (e.g., send and receive) to exchange memory addresses, and
use the one-sided writes for data transfer. As the two-sided operation is only
used for control information at the start up and in the case of failures, this procedure
does not affect performance of normal execution.
The one-sided operation for the actual
transfers makes the overall data transfer efficient. In RDMA, writes and sends
with payloads below a limit specified by devices may be written to the work
request (WR) as inlined data, thus the RNIC does not need to fetch that payload
via a DMA read. In \libname, we inline all writes whose payload is lower than
the inline limit (i.e., 64 bytes).

Normally, the RNICs actively poll a completion event (CE) from the CQ to ensure
a write resides in remote memory. Polling CE is time consuming as it involves
synchronization between the RNICs on both sides of a CQ \cite{APUS}. Thus,
for multi-group messages, we
employ \emph{selective signaling} \cite{Kalia2014} to reduce this overhead by
only checking for an CE after pushing a number of writes. When using a
selectively signaled writes with requests of size $n$, up to $n-1$ consecutive
operations can be unsignaled, i.e., a CE will not be pushed for these
operations. Note that if an operation ended with an error (e.g., a leader's
write permission is revoked), it will generate a CE even if it was posted to use
unsignaled completion.

%\paragraph{Atomic message delivery}
In a shared memory context, when a process reads entries that are
updated by another process, it is important that the reader process does not
read incomplete data that has not been fully updated by the writer process,
(e.g., processes in \libname continually monitor their shared buffer for new
messages and may be reading an incomplete entry). We resolve this issue by
adding an extra canary value at the end of each entry, as used in previous works 
\cite{APUS,FaRM,kalia2014using, islam2012high, Mu}. 
%This approach leverages the left-to-right ordering of RDMA writes. When this
%variable becomes non-zero, the ordering guarantees that the entry is complete.
Before writing a message to remote host, a process in \libname adds the checksum of the
entry to the end of the entry. A remote process always first checks
the checksum value and waits for the checksum to match the entry.


%We compared
%\libname to White-Box Atomic Multicast~\cite{gotsman2019white} in experiments
%with multiple groups and with Kernel Paxos~\cite{esposito2018kernel} in
%experiments with a single group.
%
%\subsubsection*{White-Box Atomic Multicast}
%White-Box Atomic Multicast, or WbCast, is a genuine atomic multicast protocol that can deliver multi-group messages to the leader of each destination group in three communication steps (four communication steps to the remaining replicas in the destination groups).
% WbCast provides a C-language implementation\footnote{\url{https://github.com/imdea-software/atomic-multicast}} that uses libevent for communication.
% We extended the code to split client and server and included additional statistics information.
%
%\subsubsection*{Kernel Paxos}
%Kernel Paxos is a Multi-Paxos implementation that improves the performance of the original libpaxos library~\footnote{\url{https://bitbucket.org/sciasciad/libpaxos}}.
%The main idea is to reduce system calls running Paxos logic directly into the Linux kernel and avoid the TCP/IP stack using raw sockets. 
%We used the original code\footnote{\url{https://github.com/esposem/Kernel_Paxos}} to deploy a single group with three replicas and compared the performance to \libname's.
%We have chosen such an implementation because we believe that it has similar features to our library, with optimizations for high throughput and low latency.
