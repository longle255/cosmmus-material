%!TEX root =  main.tex
\section{Implementation}
\label{sec:implementation}
This section presents implementation details and introduces competitor protocols.

We implemented a prototype of \libname in Java using the RDMA library ...(LONG, please complete here!) ... the source code is publicly available\footnote{\url{https://github.com/TODO}}. We compared \libname to White-Box Atomic Multicast~\cite{gotsman2019white} in experiments with multiple groups and with Kernel Paxos~\cite{esposito2018kernel} in experiments with a single group.

\subsubsection*{White-Box Atomic Multicast}
White-Box Atomic Multicast, or WbCast, is a genuine atomic multicast protocol that can deliver multi-group messages to the leader of each destination group in three communication steps (four communication steps to the remaining replicas in the destination groups).
 WbCast provides a C-language implementation\footnote{\url{https://github.com/imdea-software/atomic-multicast}} that uses libevent for communication.
 We extended the code to split client and server and included additional statistics information.

\subsubsection*{Kernel Paxos}
Kernel Paxos is a Multi-Paxos implementation that improves the performance of the original libpaxos library~\footnote{\url{https://bitbucket.org/sciasciad/libpaxos}}.
The main idea is to reduce system calls running Paxos logic directly into the Linux kernel and avoid the TCP/IP stack using raw sockets. 
We used the original code\footnote{\url{https://github.com/esposem/Kernel_Paxos}} to deploy a single group with three replicas and compared the performance to \libname's.
We have chosen such an implementation because we believe that it has similar features to our library, with optimizations for high throughput and low latency.
