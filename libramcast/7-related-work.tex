\section{Related work}
\label{sec:related-work}

\subsubsection*{Atomic multicast}
Atomic multicast is a well-studied problem. Skeen’s algorithm
\cite{birman1987reliable} is possibly the first atomic multicast algorithm.
Processes in Skeen assign timestamps to messages ensure that destinations agree
on the final timestamp assigned to each message, and deliver messages following
this timestamp order. The precise way in which these properties are ensured
varies from one algorithm to another. Even though it is not fault-tolerant, it
is genuine: processes only communicate if they actually are in the destinations
of the messages. Later timestamp-based genuine atomic multicast algorithms is
often implemented using a fault-tolerant version of Skeen’s protocol. Fast Paxos
\cite{coelho2017fast} optimizes the protocol by optimistically assuming a
spontaneous message ordering and deliver messages in 4$\delta$. If that
assumption is violated, Fast Paxos reverts to the basic Paxos protocol and
deliver messages in 6$\delta$. WbCast \cite{gotsman2019white} improves the
latencies of fault-tolerant version of Skeen’s protocol by combining Paxos and
Skeen’s protocol together into a single coherent protocol, and has
collision-free latency of 3$\delta$.

Ring-based protocols \cite{delporte2000fault, bezerra2015ridge,
marandi2012multi} proposed different approach to achieve high throughput by
propagating messages along a predefined ring overlay and ensure atomic multicast
properties by relying on this topology. However, ring-based algorithms are
non-genuine: involved processes communicate with processes outside of the
destination groups to deliver messages. The time complexity of these algorithms is
proportional to the number of destination groups.

\subsubsection*{RDMA systems}

Remote Direct Memory Access (RDMA) \cite{kalia2016design} is an interface that
allows servers to read and write the memory of a remote server directly. Over
the years, RDMA has become an active area of research for its high throughput,
low latency, and low CPU overhead. RDMA techniques have been implemented in
various architectures, including Infiniband \cite{pfister2001introduction}, RoCE
\cite{beck2011performance}, and iWRAP \cite{rashti200710}. RDMA has already been
explored and applied for variety of application, from key/value stores
\cite{FaRM, kalia2014using, mitchell2013using, wei2015fast}, databases
\cite{binnig2015end, huang2019rdma}, and distributed file systems
\cite{islam2012high, li2009early, wu2003pvfs}. Pilaf \cite{mitchell2013using} is
a distributed in-memory key-value store that implements client-lookup operations
with one-sided RDMA reads. In contrast, HERD \cite{kalia2014using} clients use
one-sided RDMA writes to send requests to servers which poll their receive RDMA
buffers to process requests. FaRM \cite{FaRM} proposes a distributed computing
platform, which provides the transactional interface for applications to access
the shared memory. NVFS \cite{islam2012high} provides a novel design of HDFS
with byte-addressable NVM and RDMA network. Octopus \cite{lu2017octopus} is a
distributed, shared persistent memory file system that combines RDMA and NVM's
new features by redesigning the software. Beside, many optimization guidelines
was proposed by Kalia et al. \cite{kalia2016design} to enhance performance of
RDMA system. In this work, we have applied many of the mentioned best practices
in the implementation of \libname.


\subsubsection*{RDMA-based consensus}

RDMA still has received limited attention so far in connection with consensus
protocols. Only few replication protocols enable crash-tolerant on RDMA systems
have been proposed. DARE \cite{DARE}, one of the first RDMA-based replicated
state machine introduced, aims to optimize for low latency in replica
communication. The consensus leader in DARE replicates requests to its follower
by RDMA one-side Read/Write operations, and makes use of permissions when
changing leaders. APUS \cite{APUS} improves upon DARE’s throughput. APUS
combines RDMA with Paxos and focuses on scalability with the number of
connections and replicas. APUS is based on intercepting inbound socket calls, so
it does not require modifying applications for integration. Derecho
\cite{jha2019derecho} provides an RDMA-enabled state machine replication while
structuring applications into groups, subgroups, and shards. Derecho only needs
a leader for group reconfiguration. 

However, these protocols focus on building consensus on top of RDMA primitives.
To our knowledge, there is no previous work that solved atomic multicast with
RDMA.


% \section{State machine replication}

% State machine replication (SMR) was first introduced by Lamport in \cite{Lam78}. Schneider \cite{Sch90} then presented a more systematic
% approach to the design and implementation of SMR protocols. Since then, SMR has
% become a well-known approach to replication and has been extensively studied,
% both in academia (e.g., \cite{Kapritsos:2012um, Kotla:2004ep, santos2013htsmr,
% Mencius}) and in the industry (e.g., \cite{corbett2013spanner}). SMR provides strong consistency guarantees, which
% come from total order and deterministic execution of commands. Traditional SMR
% relies on a consensus protocol to define a common order among replicas for the execution of 
% commands. Deterministic execution is usually ensured by having every replica
% execute the set of ordered commands sequentially.

% % 