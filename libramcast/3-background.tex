%!TEX root =  main.tex
\section{Background}
\label{sec:background}

In this section, we present the system model (\S\ref{sec:system-model}), define the atomic multicast communication problem (\S\ref{sec:amcast}), and quickly overview RDMA, the key enabling technology used in this work (\S\ref{sec:rdma}).

\subsection{System model}
\label{sec:system-model}

We assume a hybrid distributed system model in which processes can use both message-passing and shared-memory \cite{Aguilera2019}.
The system is composed of a set of client and server processes, which communicate by exchanging messages or accessing portions of each other's memory.
Processes are \emph{correct}, if they do not fail, or \emph{faulty}, otherwise. 
In either case, processes do not experience arbitrary behavior (i.e., no Byzantine failures).
Our protocols ensure safety under both asynchronous and synchronous execution periods. 
To ensure liveness, we assume the system is \emph{partially synchronous} \cite{DLS88}, 
that is, it is initially asynchronous and eventually becomes synchronous. The time when the 
system becomes synchronous is called the \emph{Global Stabilization Time (GST)}, and it is unknown to the processes.
Before GST, there are no bounds on communication and processing delays; after GST, such bounds exist but are unknown. 

A process can share memory regions with other processes and define permissions for those shared memory regions. 
Process $q$ can read and write a register $v$ in $p$'s memory region $mr$ with operations $\textsf{read}(p,v)$ and $\textsf{write}(p,v,value)$, respectively.
A permission associated with memory region $mr$ defines disjoint sets of processes $R_{mr}, W_{mr}, RW_{mr}$ that can read, write, or read-write the registers in region $mr$. 
Process $q$ has permission to read (respectively, write and read-write) $v$ in $p$'s $mr$ if $q \in R_{mr}$ (resp. $W_{mr}, RW_{mr}$).
A process can initially assign permissions for its shared memory regions and later change these permissions.

Processes can also communicate by exchanging messages over a set of directed links using primitives \textsf{send}$(p, m)$ and \textsf{receive}$(m)$, where $p$ is the addressee of message $m$.
We assume messages are unique.
Communication links are reliable in that every message sent by a process $p$ to another process $q$ is guaranteed to be eventually delivered by $q$ if both $p$ and $q$ are correct.
Moreover, a message is received at most once, and only if it was previously sent.


\subsection{Problem statement}
\label{sec:amcast}

Let $\Pi$ be the set of server processes in the system and $\Gamma \in 2^{\Pi}$ the set of process groups in the system, where $|\Gamma| = k$.
Groups are disjoint and each group contains $n = 2f+1$ processes, where $f$ is the maximum number of faulty processes per group.  
The assumption of disjoint groups has little practical implication since it does not prevent collocating processes that are members of different groups on the same machine.
A set of $f + 1$ processes in group $g$ is a \emph{quorum} in $g$.

A process atomically multicasts a message $m$ to groups in $m.dst$ by invoking primitive multicast($m$), where $m.dst$ is a special field in $m$ with $m$'s destinations; a process delivers $m$ with primitive deliver($m$). 
We define the relation $<$ on the set of messages processes deliver as follows: $m < m'$ iff there exists a process that delivers $m$ before $m'$. 

Atomic multicast ensures the following properties: 
\begin{itemize}
\item \textit{validity}:~if a correct process $p$ multicasts a message $m$, then eventually all correct processes $q \in g$, where $g \in m.\mathit{dst}$, r-deliver $m$.
\item \textit{integrity}:~for any process $p$ and any message $m$, $p$ r-delivers $m$ at most once, and only if $p \in g$, $g \in m.\mathit{dst}$, and $m$ was previously r-multicast.
\item \textit{uniform agreement}:~if a process $p$ a-delivers a message $m$, then eventually all correct processes $q\in m.\mathit{dst}$ a-deliver $m$.
\item \textit{uniform prefix order}:~for any two messages $m$ and $m'$ and any two processes $p$ and $q$ such that $p \in g$, $q \in h$ and $\{ g, h \} \subseteq m.\mathit{dst} \cap m'.\mathit{dst}$, if $p$ a-delivers $m$ and $q$ a-delivers $m'$, then either $p$ a-delivers $m'$ before $m$ or $q$ a-delivers $m$ before $m'$.
\item \textit{uniform acyclic order}:~the relation $<$ is acyclic.
\end{itemize}
Atomic broadcast is a special case of atomic multicast in which every message is addressed to all groups.

We require atomic multicast protocols to be \emph{genuine}~\cite{GS01b}: 
an algorithm ${\cal A}$ solving reliable or atomic multicast is genuine
if and only if for any admissible run $R$ of ${\cal A}$ and for any process $p$ in $R$, if $p$ \hl{reads a message in its buffer or writes a message remotely}, then some message $m$ is multicast, and either (a)~$p$ is the process that multicasts $m$ or (b)~$p \in g$ and $g \in m.dst$.


\subsection{Remote Direct Memory Access}
\label{sec:rdma}

%<<<<<<< HEAD
%Remote Direct Memory Access (RDMA) allows a server to access the memory of a remote server without involving the remote server's processor. 
%RDMA provides low-latency communication by bypassing the OS kernel and by implementing the network layer in hardware.
%RDMA supports message-passing operations (e.g., send, receive), shared-memory operations (e.g., read, write), and atomic operations (e.g., compare-and-swap, fetch-and-increment). 
%Previous studies have established that remote write operations provide superior performance than remote reads, and send and receive operations, and much better performance than atomic operations \cite{Kalia2014}.
%=======
Remote Direct Memory Access (RDMA) is a hardware-based protocol that enables direct data access between the memory of remote machines without the involvement of the operating system and processor. 
RDMA implements the network stack in hardware, provides both low-latency and high-bandwidth by bypassing the kernel and supporting zero-copy networking.
RDMA provides two-sided operations (e.g., send, receive), one-sided operations (e.g., read, write), and atomic operations (e.g., compare-and-swap, fetch-and-increment). The two-sided operations still have CPU involvement to remote host and user-space memory copies, thus introduces an unavoidable overhead compared to one-sided RDMA verbs \cite{FaRM}.
Besides, previous studies have established that remote write operations provide superior performance than remote reads, and send and receive operations, and much better performance than atomic operations \cite{kalia2014using, kalia2016design, mitchell2013using}.
%>>>>>>> 86c76bc4e0440ef88f14c280c5b91aba070576a1
In RamCast's normal operation, processes communicate using remote write operations only.
We refrain from using remote read operations, and resort to send and receive operations when handling failures, since they lead to simpler logic.

RDMA provides three transport modes: Reliable Connection (RC), Unreliable Connection (UC) and Unreliable Datagram (UD). 
While RC and UC are connection oriented and support only one-to-one data transmission, UD supports both one-to-one and one-to-many without establishing connections.
RC ensures data transmission is reliable and correct in the network layer, while UC does not have such a guarantee.
In this work, we use RC to provide in-order reliable delivery.
To establish a connection between two remote hosts, the RDMA-enabled network card (RNIC) on each host creates a logical RDMA endpoint known as a Queue Pair (QP), including a send queue and a receive queue for storing data transfer requests. 
Operations are posted to QPs as Work Requests (WRs) to be consumed and executed by the RNIC. 
When an RDMA operation is completed, a completion event is pushed to a Completion Queue (CQ).
Operations can be made \emph{unsignaled} by setting a flag in the WR; these verbs do not generate a completion event, and the application detects completion using application-specific methods.
Each host makes local memory regions (MR) available for remote access by asking its OS to pin the memory pages that would be used by the RNIC.
Both QPs and MRs can have different access modes (i.e., read-only or read-write). 
The hosts specify the access mode when initializing the QP or registering the MR, but the access mode can be dynamically updated later. 
The host can register the same memory for different MRs. Each MR then has its own access mode.
In this way, different remote machines can have different access rights to the same memory region. 
% The same effect can be obtained by using different access flags for the QPs used to communicate with remote machines.