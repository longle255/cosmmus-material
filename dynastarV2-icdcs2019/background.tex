%!TEX root =  main.tex
\section{Background}
\label{sec:background}


\dynastar builds on prior work on classical state machine replication and scalable state machine replication.
% (e.g.,~\cite{bezerra2014ssmr,hoang2016}) 
In this section, we briefly review these techniques. 

\subsection{State machine replication}
\label{sec:smr}

State machine replication is a fundamental approach to implementing fault-tolerant services by replicating servers, and coordinating the execution of client commands at replicas~\cite{Lam78,Sch90}. 
Every replica has a full copy of the service state, identified by a set of state variables $\vvm = \{v_1, ..., v_m\}$.
A command is a sequence of deterministic operations that can read and modify the state.

By starting in the same initial state and executing the same sequence of deterministic commands in the same order, servers make the same state changes and produce the same reply for each command. 
To guarantee that servers deliver the same sequence of commands, SMR can be implemented with atomic broadcast: commands are atomically broadcast to all servers, and all correct servers deliver and execute the same sequence of commands \cite{BJ87b,DSU04}.


\subsection{Scaling State Machine Replication}

Despite its simple execution model, classical SMR does not scale: adding resources (e.g., replicas) will not translate into sustainable improvements in throughput. 
Several approaches have been proposed to address SMR's scalability limitations.

In all proposed protocols, the idea is to shard the service's state and replicate each shard.
More precisely, the service state \vvt\ is composed of $k$ partitions, $\ppm_1, ..., \ppm_k$, where each partition $\ppm_i$ is assigned to server group $\ssm_i$. 
(For brevity, we say that server $s$ belongs to $\ppm_i$ meaning that $s \in \ssm_i$.)
%, and say ``multicast to $\ppm_i$" meaning ``multicast to server group $\ssm_i$".
%
Commands that access a single partition are executed as in classical SMR: replicas of the concerned partition agree on the execution order and each replica executes the command independently. 
Existing protocols differ in (a) how to handle multi-partition commands and (b) whether state partitioning is static or dynamic.

There are three approaches to handling multi-partition commands.
Some protocols assume a workload that can be ``perfectly partitioned," that is, there is a way to shard the service state that avoids multi-partition commands \cite{hoang2016,Nogueira17}.
Since commands are essentially single-partition, if the load is balanced across partitions then performance scales with the number of partitions.
With protocols that support commands that span multiple partitions, the multi-partition command is first ordered across the involved partitions (e.g., using an atomic multicast) and then executed by each involved partition.
Some protocols assume the workload can be sharded such that multi-partition commands are executed locally by each of the involved partitions \cite{Mu2016}, that is, the execution of a multi-partition command in one partition does not need data stored in a different partition.
For example, to support the assignment command ``$x := y$", variables $x$ and $y$ must be placed in the same partition.
However, a command that increments $x$ and increments $y$ allows these variables to be placed in different partitions.

More general solutions do not impose restrictions on state partitioning, but must incur additional overhead in the execution of multi-partition commands \cite{bezerra2014ssmr, hoang2016}.
More precisely, upon delivering a multi-partition command, replicas may need to exchange state, since some partitions may not have all the variables needed in the command.
For example, consider again the assignment command ``$x := y$" in a system where $x$ and $y$ are stored in different partitions.
With S-SMR \cite{bezerra2014ssmr}, the partitions first exchange $x$ and $y$ and then both execute the command.
DS-SMR \cite{hoang2016} supports dynamic partitioning of state; thus, either $x$ is moved to the partition that contains $y$ or vice-versa, and the command is executed at a single partition.

None of the existing approaches discuss how to partition the state of a general service in order to minimize multi-partition commands and keep the load across partitions balanced.
DS-SMR \cite{hoang2016} achieves these goals for services that can be perfectly partitioned.
In the next section, we introduce \dynastar, a protocol that minimizes multi-partition commands and balances partition load without these restrictions. 

