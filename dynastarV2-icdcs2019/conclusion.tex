\section{Conclusion}
\label{sec:conclusion}

In this paper we present \dynastar, a partitioning strategy for scalable state machine replication.
\dynastar is inspired by DS-SMR, a decentralized dynamic scheme proposed in \cite{hoang2016}.
Differently from DS-SMR, however, \dynastar performs well in all workloads evaluated.
When the state can be perfectly partitioned, \dynastar converges more quickly than DS-SMR; when partitioning cannot avoid cross-partition commands, it largely outperforms DS-SMR.
The key insights of \dynastar are to build a workload graph on-the-fly and use an optimized partitioning of the workload graph, computed with an online graph partitioner, to decide how to efficiently move state variables.
The paper describes how one can turn this conceptually simple idea into a system that sports performance close to an optimized (but impractical) scalable system.

\section*{Acknowledgements}

We wish to thank the reviewers for the insightful comments.
This work was supported in part by SNF grant numbers 175717 and 166132.