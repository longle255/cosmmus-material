%!TEX root =  main.tex
\section{System model and definitions}
\label{sec:sysmodel}

\dynastar relies on multicast abstractions to handle complex coordination among partitions without violating correctness. 
In this section, we detail the system model, define multicast, and state our correctness criterion. 

\subsection{Processes and communication}

We consider a distributed system consisting of an unbounded set of
client processes $\ccm = \{c_1, c_2, ...\}$ and a bounded set of
server processes (replicas) $\ssm = \{s_1, ..., s_n\}$.  Set $\ssm$ is
divided into disjoint groups of servers $\ssm_0, ..., \ssm_k$.
Processes are either \emph{correct}, if they never fail, or
\emph{faulty}, otherwise.  In either case, processes do not experience
arbitrary behavior (i.e., no Byzantine failures).

Processes communicate by message passing, using one-to-one or
one-to-many communication.  
One-to-one
communication uses primitives $send(p,m)$ and $receive(m)$, where $m$
is a message and $p$ is the process $m$ is addressed to.  If sender
and receiver are correct, then every message sent is eventually
received.
One-to-many communication relies on atomic
multicast, defined in \S\ref{sec:amcast}.

The system is \emph{partially synchronous}~\cite{DLS88}: it is initially asynchronous and eventually becomes synchronous. 
When the system is asynchronous, there are no bounds on the time it takes for messages to be transmitted and actions to be executed; when the system  is synchronous, such bounds exist but are unknown to the processes.
%The partially synchronous assumption allows consensus, a fundamental problem at the core of replication~\cite{Lam98,Sch90}, to be implemented under realistic conditions~\cite{FLP85,Lam98}.

\subsection{Atomic multicast}
\label{sec:amcast}

To atomically multicast a message $m$ to a set of groups $\gamma$,
processes use primitive \amcast$(\gamma, m)$.  Message $m$ is
delivered at the destinations with \amdel$(m)$.  We define delivery
order $<$ as follows: $m < m'$ iff there exists a process that
delivers $m$ before $m'$.

Atomic multicast ensures the following properties:

\begin{itemize}
    
    \item[--] If a correct process \amcast{}s $m$, every correct
      process in a group in $\gamma$ \amdel{}s $m$ \emph{(validity)}.
    
    \item[--] If a process \amdel{}s $m$, then every correct process
      in a group in $\gamma$ \amdel{}s $m$ \emph{(uniform agreement)}.
    
    \item[--] For any message $m$, every process \amdel{}s $m$ at most
      once, and only if some process has \amcast{} $m$ previously
      \emph{(integrity)}.
    
    \item[--] If a process \amcast{}s $m$ and then $m'$, then no process \amdel{}s $m'$ before $m$ \emph{(fifo order)}.

    \item[--] The delivery order is acyclic \emph{(atomic order)}.

    \item[--] For any messages $m$ and $m'$ and any processes $p$ and
      $q$ such that $p \in g$, $q \in h$ and $\{ g, h \} \subseteq
      \gamma$, if $p$ delivers $m$ and $q$ delivers $m'$, then either
      $p$ delivers $m'$ before $m$ or $q$ delivers $m$ before $m'$
      \emph{(prefix order)}.
    
\end{itemize}

Atomic broadcast is a special case of atomic multicast in which there
is a single group of processes.

\subsection{Correctness criterion}
\label{sec:correctcrit}

DynaStar ensures linearizable executions.  An execution is
\emph{linearizable} if there is a way to reorder the client commands
in a sequence that (i)~respects the semantics of the commands, as
defined in their sequential specifications, and (ii)~respects the
real-time precedence of commands~\cite{Attiya04}.


