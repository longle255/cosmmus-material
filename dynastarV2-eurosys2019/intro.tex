%!TEX root =  main.tex
\section{Introduction}

Modern distributed systems typically scale performance by sharding application data and tolerate failures by replicating each shard. 
Clients submit commands for execution to one or more shards.
Within a shard, replicas coordinate by means of a consensus protocol (e.g., Paxos~\cite{Lam98}).
To coordinate the execution of multi-shard commands, replicated shards rely on some distributed coordination protocol (e.g., two-phase locking \cite{corbett2013spanner}, optimistic concurrency control \cite{Chang:2008}, atomic multicast \cite{bezerra2014ssmr}).

Even though different techniques have been proposed to handle commands that access state in multiple shards, inherently multi-shard commands are more expensive than single-shard commands.
This happens due to overhead from ordering and coordinating commands across shards to ensure strong consistency.
Moreover, if data is not distributed carefully, then load imbalances can nullify the benefits of sharding. 
Consequently, the achilles heel of sharded systems lies on the partitioning of the service state.
An ideal partitioning scheme is one that would both (i) allow commands to be executed at a single partition only, and (ii) evenly distribute data so that load is balanced among partitions.
Under such conditions, performance will scale with the number of partitions.


%State machine replication (SMR) is a fundamental technique for building fault-tolerant systems and services. 
%With SMR, state is replicated on a set of servers and each replica deterministically executes the same sequence of client commands in order to maintain consistency~\cite{Lam78,Sch90}. 
%State machine replication provides configurable fault tolerance (i.e., by setting the number of replicas) but limited performance since all replicas must execute all commands.
%%
%In order to provide both configurable fault tolerance and scalable performance, some recent proposals have extended state machine replication with sharding (e.g., \cite{Aguilera:2007, bezerra2014ssmr,corbett2013spanner,Glendenning:2011kj,
%  bli16edcc}).
%Essentially, these approaches partition (shard) the service state and replicate each partition.
%Commands that access state in a single partition are handled as in classic state machine replication.
%Different techniques have been proposed to handle commands that access state in multiple partitions, but inherently multi-partition commands are more expensive than single-partition commands.
%This happens due to overhead from ordering and coordinating commands across partitions to ensure strong consistency.
%Moreover, if data is not distributed carefully, then load imbalances can nullify the benefits of partitioning. 
%Consequently, the achilles heel of scalable state machine replication lies on the partitioning of the service state.
%An ideal partitioning scheme is one that would both (i) allow commands to be executed at a single partition only, and (ii) evenly distribute data so that load is balanced among partitions.
%Under such conditions, performance will scale with the number of partitions.

Partitioning the service state in order to achieve scalable performance, however, is challenging.
%This happens for two main reasons.
First, it depends on a good understanding of the workload, which may not be available.
Second, even if enough information is available, finding a good partitioning is a complex optimization problem~\cite{curino2010sch,taft2014est}. 
Third, workloads may vary over time. 
In social networks, for example, some users may experience a surge increase in their number of followers (e.g., new ``celebrities''); workload demand may shift along the hours of the day and the days of the week; and unexpected (e.g., a video that goes viral) or planned events (e.g., a new company starts trading in the stock exchange) may lead to exceptional periods when requests increase significantly higher than in normal periods.
These challenges perhaps explain why most approaches that rely on state partitioning delegate the task of partitioning the service state to the application designer (e.g., \cite{Aguilera:2007, bezerra2014ssmr,corbett2013spanner,bli16edcc}).

In this paper, we introduce \dynastar, a new approach to scalable replication.
\dynastar differs from previous solutions in that it supports \emph{dynamic state partitioning}.
This means that data can be relocated from one shard (or partition) to another to optimize performance.
Moreover, data relocation is done on-the-fly, with minimal service disruption.
As a result, \dynastar does not require any a priori information about the workload and can adapt to workload variations.
To achieve this design, \dynastar maintains a
location oracle with a global view of the application state.  The oracle minimizes the
number of state relocations by monitoring the workload, and
re-computing an optimized partitioning on demand using a 
partitioning algorithm (METIS).  
The location oracle maintains two data structures: (i) a mapping of application state variables to shards, and (ii) a \emph{workload graph} with state variables as vertices and edges as commands that access the variables.  
Before a client submits a command, it contacts the
location oracle to discover the shards on which state variables are
stored.  If the command accesses variables in multiple shards, the
client chooses one shard to execute the command and instructs the other involved shards to temporarily relocate 
the needed variables to the chosen shard. Of course, when relocating a variable, the client
is faced with a choice of which shard to use as a destination.
The client estimates the partition that minimizes the number of state relocations.
% using the METIS partitioner.
%Although METIS does not necessarily produce the best possible partitioning of the workload graph, it offers a good compromise between performance and partitioning quality.

To tolerate failures, \dynastar implements the oracle as a regular partition, replicated in a set of nodes.
To ensure that the oracle does not become a performance bottleneck, clients cache location information.
Therefore, clients only query the oracle when they access a variable for the first time or when cached entries become invalid (i.e., because a variable changed location).
\dynastar copes with commands addressed to wrong partitions by telling clients to retry a command.

We implemented \dynastar and compared its performance to
alternative schemes, including an idealized approach that knows the workload ahead of time.
Although this scheme is not achievable in practice, it provides an interesting baseline to compare against.
\dynastar's performance rivals that of the idealized scheme, while having no a priori knowledge of the workload.
In most workloads of interest, \dynastar largely outperforms existing dynamic schemes.
We also show that the location oracle never becomes a bottleneck in our experiments and can handle workload graphs with millions of vertices.

The paper makes the following contributions:
\begin{itemize}
\item It introduces \dynastar and discusses its implementation. 
\item It evaluates different partitioning schemes for state machine replication under a variety of conditions.
\item It presents a detailed experimental evaluation of \dynastar using the TPC-C benchmark and a social network service.

%including real social network graphs with half a million users and 14 million edges.
\end{itemize}

The rest of the paper is structured as follows.
Section~\ref{sec:sysmodel} presents the system model.
Section~\ref{sec:background} reviews existing scalable state machine replication approaches.
Section~\ref{sec:dynastar} introduces \dynastar, describes the technique in detail and argues about its correctness.
Section~\ref{sec:implementation} overviews our prototype.
Section~\ref{sec:experiments} reports on the results of our experiments.
Section~\ref{sec:rw} surveys related work and
Section~\ref{sec:conclusion} concludes the paper.
