\section{Background and motivation}
State-machine replication is a fundamental approach to implement a fault-tolerant service by replicating servers and coordinating the execution of client commands against server replicas~\cite{Lam78,Sch90}. 
State-machine replication ensures strong consistency (i.e., linearizability~\cite{Attiya04}) by coordinating the execution of commands in the different replicas: Every replica has a full copy of the service state $\vvm = \{v_1, ..., v_m\}$ and executes commands submitted by the clients in the same order. A command is a program consisting of a sequence of operations, which can be of three types: \emph{read(v)}, \emph{write(v, val)}, or a deterministic computation.

\subsection{Scaling state-machine replication}
By starting in the same initial state and executing the same sequence of deterministic commands, servers make the same state changes and produce the same reply for each command. To guarantee that servers deliver the same sequence of commands, SMR can be implemented with atomic broadcast: commands are atomically broadcast to all servers, and all correct servers deliver and execute the same sequence of commands \cite{BJ87b,DSU04}.

Despite its simple execution model, classical SMR does not scale: adding resources (e.g., replicas) will not translate into sustainable improvements in throughput. This happens for a few reasons. First, the underlying communication protocol needed to ensure ordered message delivery may not scale itself (i.e., a communication bottleneck). Second, every command must be executed sequentially by each replica (i.e., an execution bottleneck).

Several approaches have been proposed to address SMR’s scalability limitations. To cope with communication overhead, some proposals have suggested to spread the load of ordering commands among multiple processes (e.g., \cite{Moraru:2013gw,Mencius,Marandi:2012hb}), as opposed to dedicating a single process to determine the order of commands (e.g., \cite{CT96,Lamport:1998ea}).

Two directions of research have been suggested to overcome execution bottlenecks. One approach (scaling up) is to take advantage of multiple cores to execute commands concurrently without sacrificing consistency \cite{Kapritsos:2012um,Marandi:2014bj,Kotla:2004ep,Guo:2014jp}. Another approach (scaling out) is to partition the service's state and replicate each partition (e.g., \cite{Glendenning:2011kj,Marandi:2011dj}. In the following section, we review Scalable State-Machine Replication (S-SMR), a proposal in the second category.

\subsection{Scalable State-Machine Replication}
TODO: talk about SSMR

SSMR implementation brings into use the concept of \emph{Oracle}, the core of partitioning algorithms, which runs a static deterministic algorithm to return the combination of involved partitions of a command; all clients and partitions have their own version of Oracle, and assumed to be identical. By doing that way, SSMR can ensure Oracle return same results for query from both clients and partitions. However, this implementation leads to some limitations: (i) the Oracles on all parties are not synchronized, thus they need to have the knowledge of all $v \in \vvm$ during the life-cycle of the system, and (ii) a change in $\vvm$ on one Oracle will not be recognized by the others. Therefore, SSMR doesn't support creating state variable $v_n$ on the fly, and has to initialize the whole $\vvm$ on the starting phase.
